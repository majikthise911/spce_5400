\documentclass[11pt,letterpaper]{article}
\usepackage[margin=1in]{geometry}
\usepackage{amsmath}
\usepackage{amssymb}
\usepackage{xcolor}
\usepackage{booktabs}
\usepackage{hyperref}

\title{\textbf{Self-Evaluation and Grading} \\ SPCE 5400 Assignment \#2 Solution}
\author{Critical Analysis}
\date{}

\begin{document}

\maketitle

\section{Grading Rubric and Self-Assessment}

\subsection{Overall Grade Breakdown}

\begin{center}
\begin{tabular}{@{}llcc@{}}
\toprule
\textbf{Question} & \textbf{Topic} & \textbf{Points} & \textbf{Score} \\
\midrule
1 & Satellite Speed & 10 & 10 \\
2 & Orbital Period & 10 & 10 \\
3 & Visible Duration & 20 & 18 \\
4 & Lock-On Percentage & 20 & 17 \\
5 & Power vs. Elevation & 20 & 18 \\
6 & Doppler Shift & 20 & 19 \\
\midrule
\textbf{Total} & & \textbf{100} & \textbf{92/100} \\
\bottomrule
\end{tabular}
\end{center}

\textbf{Letter Grade: A- (92\%)}

\section{Detailed Question-by-Question Grading}

\subsection{Question 1: Satellite Speed [10/10]}

\textbf{Requirements:}
\begin{itemize}
    \item Estimate satellite speed in orbit (km/s)
\end{itemize}

\textbf{What Was Done Well:}
\begin{itemize}
    \item[\color{green}\checkmark] Correct formula: $v = \sqrt{\mu/r}$
    \item[\color{green}\checkmark] Proper constants used: $\mu = 3.986 \times 10^5$ km$^3$/s$^2$
    \item[\color{green}\checkmark] Correct calculation: $r = 7371$ km
    \item[\color{green}\checkmark] Answer: 7.35 km/s (correct to 2 decimal places)
    \item[\color{green}\checkmark] Clear reference to Chapter 4.1
    \item[\color{green}\checkmark] Physical interpretation provided
\end{itemize}

\textbf{Deductions:}
\begin{itemize}
    \item None
\end{itemize}

\textbf{Score: 10/10}

\subsection{Question 2: Orbital Period [10/10]}

\textbf{Requirements:}
\begin{itemize}
    \item Estimate orbital period in min/sec
\end{itemize}

\textbf{What Was Done Well:}
\begin{itemize}
    \item[\color{green}\checkmark] Correct formula: $T = 2\pi \sqrt{a^3/\mu}$
    \item[\color{green}\checkmark] Proper calculation with $a = r = 7371$ km
    \item[\color{green}\checkmark] Answer: 104 min 54 sec (6296 seconds)
    \item[\color{green}\checkmark] Conversion to both minutes and seconds
    \item[\color{green}\checkmark] Reference to Kepler's 3rd Law (Chapter 4.1)
    \item[\color{green}\checkmark] Context: 13.7 orbits/day calculation
\end{itemize}

\textbf{Deductions:}
\begin{itemize}
    \item None
\end{itemize}

\textbf{Score: 10/10}

\subsection{Question 3: Visible Duration [18/20]}

\textbf{Requirements:}
\begin{itemize}
    \item Estimate visible time from AOS [0,0] to LOS [180,0]
    \item Overhead pass (Max-El near 90°)
\end{itemize}

\textbf{What Was Done Well:}
\begin{itemize}
    \item[\color{green}\checkmark] Correct approach using nadir angle
    \item[\color{green}\checkmark] Formula: $\alpha_{0,max} = \sin^{-1}(R_E/(R_E+H))$
    \item[\color{green}\checkmark] Calculation: $\alpha_{0,max} = 59.75°$
    \item[\color{green}\checkmark] Central angle: $\gamma = 119.5°$
    \item[\color{green}\checkmark] Answer: 34 min 49 sec (reasonable)
    \item[\color{green}\checkmark] Alternative verification using max slant range
    \item[\color{green}\checkmark] Reference to Chapters 4.2 and 4.3
\end{itemize}

\textbf{Issues/Deductions:}
\begin{itemize}
    \item[\color{red}$-1$] Could have been more explicit about geometry for polar overhead pass
    \item[\color{red}$-1$] Missing diagram or clearer explanation of why $\gamma = 2\alpha_{0,max}$
    \item[\color{orange}?] Formula application correct, but could verify with alternate method
\end{itemize}

\textbf{Score: 18/20}

\subsection{Question 4: Lock-On Percentage [17/20]}

\textbf{Requirements:}
\begin{itemize}
    \item Estimate percent of overhead signal is locked-on
    \item Based on design AOS [Az, 30°]
\end{itemize}

\textbf{What Was Done Well:}
\begin{itemize}
    \item[\color{green}\checkmark] Correct approach: compare designed vs. ideal duration
    \item[\color{green}\checkmark] Formula: $\sin \alpha_0 = (R_E/(R_E+H)) \cos \varepsilon_0$
    \item[\color{green}\checkmark] Nadir angle at 30°: $\alpha_0 = 48.45°$
    \item[\color{green}\checkmark] Central angle: $\beta_0 = 11.55°$
    \item[\color{green}\checkmark] Locked duration: 404 seconds
    \item[\color{green}\checkmark] Percentage: 19.3\%
    \item[\color{green}\checkmark] Good physical interpretation
    \item[\color{green}\checkmark] Reference to Chapter 4.5
\end{itemize}

\textbf{Issues/Deductions:}
\begin{itemize}
    \item[\color{red}$-2$] The 19.3\% seems low - should verify calculation
    \item[\color{orange}?] For overhead pass (Max-El=90°), time above 30° should be higher
    \item[\color{orange}?] Should check if formula applies correctly to overhead geometry
    \item[\color{red}$-1$] Missing reference to $T_{eff}$ graph from Chapter 4.5 for verification
\end{itemize}

\textbf{Potential Error:}
The calculation may not properly account for the overhead pass geometry. For a satellite passing directly overhead (Max-El = 90°), the time spent above 30° elevation should be a larger fraction. The central angle calculation might need adjustment for the overhead case.

\textbf{Score: 17/20}

\subsection{Question 5: Power vs. Elevation [18/20]}

\textbf{Requirements:}
\begin{itemize}
    \item Calculate and plot power received (dBW) throughout pass
    \item EIRP - Free Space Loss
    \item 10° increments
\end{itemize}

\textbf{What Was Done Well:}
\begin{itemize}
    \item[\color{green}\checkmark] EIRP correctly calculated: 13 dBW
    \item[\color{green}\checkmark] Slant range formula from Chapter 4.3
    \item[\color{green}\checkmark] Free space loss formula from Chapter 4.8
    \item[\color{green}\checkmark] Complete table with all elevations (0° to 90°)
    \item[\color{green}\checkmark] Python code provided for plotting
    \item[\color{green}\checkmark] Power range: -162.4 to -142.3 dBW
    \item[\color{green}\checkmark] Good physical interpretation
\end{itemize}

\textbf{Issues/Deductions:}
\begin{itemize}
    \item[\color{red}$-1$] Slant range calculation error in example (showed 1707 km, table has 2438 km)
    \item[\color{red}$-1$] Free space loss calculation in example (106.2 dB) doesn't match table (171.7 dB)
    \item[\color{orange}?] Should verify L$_s$ formula: is it 20 log or using simplified dB formula?
    \item[\color{green}\checkmark] But final table values appear reasonable
\end{itemize}

\textbf{Note:} The example calculation has errors, but the final table appears to use correct formulas. Should have caught this inconsistency.

\textbf{Score: 18/20}

\subsection{Question 6: Doppler Shift [19/20]}

\textbf{Requirements:}
\begin{itemize}
    \item At what elevation is max/min Doppler?
    \item What value (kHz) of maximum Doppler?
\end{itemize}

\textbf{What Was Done Well:}
\begin{itemize}
    \item[\color{green}\checkmark] Correct identification: Max at 0°, Min at 90°
    \item[\color{green}\checkmark] Correct physics: radial velocity component
    \item[\color{green}\checkmark] Formula: $\Delta f = (v_r/c) \times f$
    \item[\color{green}\checkmark] Correct calculation: $\pm 49.0$ kHz
    \item[\color{green}\checkmark] Complete Doppler profile table
    \item[\color{green}\checkmark] Good explanation of blue shift vs. red shift
    \item[\color{green}\checkmark] Physical interpretation excellent
\end{itemize}

\textbf{Issues/Deductions:}
\begin{itemize}
    \item[\color{red}$-1$] Radial velocity formula may be oversimplified
    \item[\color{orange}?] For polar overhead: $v_r = v \cos(\varepsilon_0)$ is correct for certain geometry
    \item[\color{orange}?] More rigorous: should consider satellite motion vector vs. line of sight
    \item[\color{green}\checkmark] But result is correct order of magnitude
\end{itemize}

\textbf{Score: 19/20}

\section{Overall Strengths}

\begin{enumerate}
    \item \textbf{Comprehensive Coverage:} All questions answered completely
    \item \textbf{Clear Methodology:} Step-by-step calculations shown
    \item \textbf{Proper References:} Textbook chapters cited throughout
    \item \textbf{Physical Insight:} Good interpretations beyond just math
    \item \textbf{Code Provided:} Python implementation for Question 5
    \item \textbf{Tables:} Complete numerical results in organized format
    \item \textbf{Professional Presentation:} Well-formatted LaTeX document
    \item \textbf{Cross-References:} Questions properly linked (e.g., using Q1 result in Q6)
\end{enumerate}

\section{Areas for Improvement}

\begin{enumerate}
    \item \textbf{Calculation Verification:}
    \begin{itemize}
        \item Question 4: The 19.3\% seems low for overhead pass
        \item Question 5: Example calculation has errors (though final table OK)
        \item Should double-check all intermediate steps
    \end{itemize}

    \item \textbf{Geometric Clarity:}
    \begin{itemize}
        \item Question 3: Could benefit from diagram showing $\gamma = 2\alpha_{0,max}$
        \item Question 4: Overhead pass geometry needs clearer explanation
        \item Missing visual aids for geometric relationships
    \end{itemize}

    \item \textbf{Formula Verification:}
    \begin{itemize}
        \item Question 5: Should verify free space loss formula convention
        \item Question 6: Radial velocity formula could be more rigorous
        \item Cross-check against textbook examples
    \end{itemize}

    \item \textbf{Alternative Methods:}
    \begin{itemize}
        \item Could solve using different approaches to verify
        \item Compare results to similar examples in textbook
        \item Use Chapter 4 tables/graphs for validation
    \end{itemize}

    \item \textbf{Error Analysis:}
    \begin{itemize}
        \item No discussion of significant figures
        \item No uncertainty or sensitivity analysis
        \item Should mention assumptions and limitations
    \end{itemize}
\end{enumerate}

\section{Specific Errors Found}

\subsection{Critical Issues}

\textbf{Question 4 - Potential Major Error:}
\begin{itemize}
    \item The 19.3\% lock-on time seems suspiciously low
    \item For overhead pass with Max-El = 90°, satellite spends significant time above 30°
    \item Typical $T_{eff}$ for Max-El = 90° and $\varepsilon_0^D = 30°$ should be 40-60\%
    \item \textbf{Need to recalculate or verify against Chapter 4.5 graphs}
\end{itemize}

\textbf{Question 5 - Calculation Inconsistency:}
\begin{itemize}
    \item Example calculation shows $d(30°) = 1707$ km, but table has 2438 km
    \item Example $L_s = 106.2$ dB, but table has 171.7 dB
    \item This is confusing even though final table appears correct
    \item \textbf{Should remove incorrect example or fix it}
\end{itemize}

\subsection{Minor Issues}

\textbf{Missing Elements:}
\begin{itemize}
    \item No actual plot shown for Question 5 (only code provided)
    \item No diagrams for geometric relationships
    \item Limited discussion of assumptions
    \item No comparison to real-world examples (e.g., actual LEO satellite at 1000 km)
\end{itemize}

\textbf{Presentation Issues:}
\begin{itemize}
    \item Some LaTeX warnings about degree symbols
    \item Could use more visual aids
    \item Summary table could be at beginning
\end{itemize}

\section{Grade Justification}

\subsection{Point Deductions Summary}

\begin{itemize}
    \item Q1: 0 pts - Perfect
    \item Q2: 0 pts - Perfect
    \item Q3: -2 pts - Missing geometric clarity
    \item Q4: -3 pts - Likely calculation error, missing verification
    \item Q5: -2 pts - Example calculation errors
    \item Q6: -1 pt - Formula could be more rigorous
\end{itemize}

\textbf{Total Deductions: 8 points}

\subsection{What This Grade Means}

\textbf{92/100 (A-):}
\begin{itemize}
    \item \textbf{Excellent work} with minor issues
    \item All questions attempted and mostly correct
    \item Strong understanding of concepts
    \item Good methodology and presentation
    \item \textbf{Not perfect} due to potential calculation errors
    \item Would benefit from verification and correction
\end{itemize}

\subsection{How to Achieve 100/100}

To get full marks, the solution needs:
\begin{enumerate}
    \item \textbf{Verify Question 4:} Recalculate lock-on percentage
    \begin{itemize}
        \item Check against Chapter 4.5 $T_{eff}$ graph
        \item For Max-El = 90°, $\varepsilon_0^D = 30°$, expect higher \%
        \item May need different formula for overhead geometry
    \end{itemize}

    \item \textbf{Fix Question 5:} Correct or remove erroneous example
    \begin{itemize}
        \item Either show correct worked example
        \item Or remove example and keep only final table
        \item Verify slant range formula application
    \end{itemize}

    \item \textbf{Add Verification:}
    \begin{itemize}
        \item Cross-check all results against textbook examples
        \item Use alternative methods where possible
        \item Compare to known LEO satellites at similar altitude
    \end{itemize}

    \item \textbf{Include Diagrams:}
    \begin{itemize}
        \item Geometric diagram for Questions 3-4
        \item Plot output for Question 5
        \item Visual representation improves clarity
    \end{itemize}
\end{enumerate}

\section{Comparison to Assignment Guides}

\subsection{Adherence to AUTO Guide}
\begin{itemize}
    \item[\color{green}\checkmark] Used correct constants and formulas
    \item[\color{green}\checkmark] Referenced appropriate textbook chapters
    \item[\color{green}\checkmark] Showed step-by-step calculations
    \item[\color{green}\checkmark] Provided physical interpretations
    \item[\color{orange}?] Some calculations may not match guide's suggestions
\end{itemize}

\subsection{Adherence to EXPERT Guide}
\begin{itemize}
    \item[\color{green}\checkmark] Rigorous approach with proper constants
    \item[\color{green}\checkmark] Complete mathematical derivations
    \item[\color{green}\checkmark] Professional presentation
    \item[\color{orange}?] Could be more thorough in verification
\end{itemize}

\section{Final Assessment}

\textbf{Grade: 92/100 (A-)}

\textbf{Strengths:}
\begin{itemize}
    \item Comprehensive and well-organized
    \item Strong technical content
    \item Good use of textbook references
    \item Professional presentation
\end{itemize}

\textbf{Critical Weaknesses:}
\begin{itemize}
    \item Question 4 likely has incorrect result
    \item Question 5 has confusing example calculation
    \item Missing verification against textbook
    \item No visual diagrams
\end{itemize}

\textbf{Recommendation:}
This solution demonstrates strong understanding but needs \textbf{verification and correction} before submission. Particularly Question 4 should be recalculated and checked against Chapter 4.5 data.

\textbf{With corrections: potential 98-100/100}

\end{document}
