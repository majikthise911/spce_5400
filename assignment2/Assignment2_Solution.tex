\documentclass[11pt,letterpaper]{article}
\usepackage[margin=1in]{geometry}
\usepackage{amsmath}
\usepackage{amssymb}
\usepackage{hyperref}
\usepackage{graphicx}
\usepackage{listings}
\usepackage{xcolor}
\usepackage{booktabs}

\title{\textbf{SPCE 5400 Assignment \#2} \\ Ground Station Analysis - Complete Solution}
\author{Topics: Chapters 4 \& 5}
\date{Due: October 15, 2025}

\begin{document}

\maketitle

\section{Problem Statement}

Given a satellite in circular, polar LEO at an altitude of 1000 km, with the following parameters:

\textbf{Satellite Tx Signal:}
\begin{itemize}
    \item Tx power: 10 W (10 dBW)
    \item Frequency: 2.0 GHz
    \item Satellite antenna Gain: 3 dBi
    \item Atmospheric Conditions: No rain, no adverse weather
\end{itemize}

\textbf{Ground Station:}
\begin{itemize}
    \item Location: Pikes Peak, CO [no visible obstructions]
    \item Lock-on Design: AOS [Az, 30°]
\end{itemize}

\section{Constants and Key Equations}

\subsection{Standard Constants}
From Chapter 4.1 and standard references:
\begin{itemize}
    \item Earth radius: $R_E = 6371$ km
    \item Gravitational parameter: $\mu = 3.986 \times 10^5$ km$^3$/s$^2$
    \item Speed of light: $c = 3 \times 10^5$ km/s (or $3 \times 10^8$ m/s)
    \item Frequency: $f = 2.0$ GHz = $2 \times 10^9$ Hz
\end{itemize}

\subsection{Orbital Parameters}
\begin{itemize}
    \item Altitude: $H = 1000$ km
    \item Orbital radius: $r = R_E + H = 6371 + 1000 = 7371$ km
    \item Inclination: $i = 90°$ (polar orbit)
    \item Eccentricity: $e = 0$ (circular orbit)
\end{itemize}

\section{Question 1: Satellite Speed in Orbit [10 pts]}

\subsection{Reference Material}
\textbf{Textbook: Chapter 4.1 - LEO Satellite Tracking Principles}

The orbital velocity for circular orbits is derived from balancing gravitational and centripetal forces.

\subsection{Formula and Its Source}

\textbf{Source:} Chapter 4.1 (standard circular orbit velocity from orbital mechanics)
\begin{equation}
v = \sqrt{\frac{\mu}{r}}
\end{equation}

\textbf{What this means:} For a satellite in circular orbit, the gravitational force pulling it toward Earth exactly balances the centripetal force needed to maintain circular motion. This balance determines a unique orbital velocity for each altitude.

\textbf{Derivation from physics:}
\begin{enumerate}
    \item Gravitational force: $F_g = \frac{\mu m}{r^2}$ (where $m$ is satellite mass)
    \item Centripetal force needed for circular motion: $F_c = \frac{m v^2}{r}$
    \item Setting them equal: $\frac{\mu m}{r^2} = \frac{m v^2}{r}$
    \item Canceling mass and solving for $v$: $v = \sqrt{\frac{\mu}{r}}$
\end{enumerate}

\textbf{Physical insight:}
\begin{itemize}
    \item Higher orbits (larger $r$) have lower velocities (Kepler's laws)
    \item Lower orbits (smaller $r$) require higher velocities to avoid falling
    \item At 1000 km altitude, $r = R_E + H = 6371 + 1000 = 7371$ km
\end{itemize}

\subsection{Solution}

\textbf{Overview:} We need to find the orbital speed of a satellite at 1000 km altitude in circular orbit.

\textbf{Strategy:}
\begin{enumerate}
    \item Calculate the orbital radius $r = R_E + H$
    \item Apply the circular orbit velocity formula
    \item Perform the calculation step-by-step
\end{enumerate}

\vspace{0.3cm}

\textbf{Step 1: Identify the given values}

\textbf{From the problem statement:}
\begin{itemize}
    \item Altitude: $H = 1000$ km
    \item Earth radius: $R_E = 6371$ km (standard constant)
    \item Earth's gravitational parameter: $\mu = 3.986 \times 10^5$ km$^3$/s$^2$ (standard constant)
\end{itemize}

\vspace{0.3cm}

\textbf{Step 2: Calculate orbital radius}

\textbf{Why we do this:} The velocity formula requires the distance from Earth's center, not just the altitude above the surface.

\textbf{Formula:}
\begin{equation}
r = R_E + H
\end{equation}

\textbf{Calculation:}
\begin{align}
r &= 6371 \text{ km} + 1000 \text{ km} \\
r &= 7371 \text{ km}
\end{align}

\textbf{What this means:} The satellite orbits at 7371 km from Earth's center.

\vspace{0.3cm}

\textbf{Step 3: Apply the velocity formula}

\textbf{Why we do this:} This gives us the speed needed to maintain circular orbit at this radius.

\textbf{Formula:} Chapter 4.1:
\begin{equation}
v = \sqrt{\frac{\mu}{r}}
\end{equation}

\textbf{Substitute the values:}
\begin{equation}
v = \sqrt{\frac{3.986 \times 10^5 \text{ km}^3\text{/s}^2}{7371 \text{ km}}}
\end{equation}

\textbf{Divide inside the square root:}
\begin{equation}
\frac{3.986 \times 10^5}{7371} = 54.079 \text{ km}^2\text{/s}^2
\end{equation}

\textbf{What we're calculating:} This intermediate value has units of (velocity)$^2$, so taking the square root gives velocity.

\textbf{Take the square root:}
\begin{align}
v &= \sqrt{54.079 \text{ km}^2\text{/s}^2} \\
v &= 7.354 \text{ km/s}
\end{align}

\textbf{Rounding to 3 significant figures:}
\begin{equation}
v = 7.35 \text{ km/s}
\end{equation}

\vspace{0.3cm}

\textbf{Answer: The satellite orbital speed is 7.35 km/s}

\subsection{Physical Interpretation}
This speed is typical for LEO satellites. For reference:
\begin{itemize}
    \item ISS (400 km): $\approx$ 7.66 km/s (lower orbit, higher speed)
    \item Our satellite (1000 km): 7.35 km/s
    \item GPS satellites (20,200 km): $\approx$ 3.87 km/s (higher orbit, lower speed)
    \item This confirms Kepler's observation: higher orbits have lower velocities
\end{itemize}

\subsection{Verification}
\textbf{Reasonableness check:}
\begin{itemize}
    \item 7.35 km/s = 26,460 km/h = 16,440 mph
    \item This is much faster than any aircraft but slower than escape velocity (11.2 km/s)
    \item At this speed, the satellite travels its entire 46,300 km orbit in about 105 minutes (verified in Question 2)
\end{itemize}

\section{Question 2: Satellite Orbit Period [10 pts]}

\subsection{Reference Material}
\textbf{Textbook: Chapter 4.1 - Orbital Period from Kepler's Third Law}

For circular orbits, the semi-major axis $a = r$ (orbital radius).

\subsection{Formula and Its Source}

\textbf{Source:} Chapter 4.1 (Kepler's Third Law for circular orbits)
\begin{equation}
T = 2\pi \sqrt{\frac{a^3}{\mu}} = 2\pi \sqrt{\frac{r^3}{\mu}}
\end{equation}

\textbf{What this means:} The orbital period (time for one complete orbit) depends only on the orbital radius. Larger orbits take longer to complete.

\textbf{Historical context:} Kepler's Third Law (1619) states that the square of the period is proportional to the cube of the semi-major axis: $T^2 \propto a^3$. When expressed with Earth's gravitational parameter $\mu$, this becomes the formula above.

\textbf{Physical insight:}
\begin{itemize}
    \item For circular orbits, semi-major axis $a$ equals orbital radius $r$
    \item The $2\pi$ comes from the circular path (circumference = $2\pi r$)
    \item Higher orbits: larger circumference AND slower speed $\Rightarrow$ much longer period
    \item The relationship is non-linear: double the radius gives $2^{3/2} = 2.83$ times longer period
\end{itemize}

\subsection{Solution}

\textbf{Overview:} We need to find how long it takes the satellite to complete one full orbit around Earth.

\textbf{Strategy:}
\begin{enumerate}
    \item Use the orbital radius from Question 1: $r = 7371$ km
    \item Apply Kepler's Third Law formula
    \item Perform the calculation step-by-step, showing all arithmetic
    \item Convert the result to minutes and seconds
\end{enumerate}

\vspace{0.3cm}

\textbf{Step 1: Identify the given values}

\textbf{From Question 1:}
\begin{itemize}
    \item Orbital radius: $r = 7371$ km
    \item Earth's gravitational parameter: $\mu = 3.986 \times 10^5$ km$^3$/s$^2$
\end{itemize}

\vspace{0.3cm}

\textbf{Step 2: Apply Kepler's Third Law}

\textbf{Why we do this:} This formula directly gives us the period for any circular orbit.

\textbf{Formula:} Chapter 4.1:
\begin{equation}
T = 2\pi \sqrt{\frac{r^3}{\mu}}
\end{equation}

\textbf{Substitute the values:}
\begin{equation}
T = 2\pi \sqrt{\frac{(7371)^3}{3.986 \times 10^5}}
\end{equation}

\vspace{0.3cm}

\textbf{Step 3: Calculate $r^3$ (cube the orbital radius)}

\textbf{Why we do this:} The formula requires $r^3$, not just $r$.

\textbf{First, square the radius:}
\begin{align}
r^2 &= (7371)^2 \\
r^2 &= 54,331,641 \text{ km}^2
\end{align}

\textbf{Then multiply by $r$ again:}
\begin{align}
r^3 &= r^2 \times r \\
r^3 &= 54,331,641 \times 7371 \\
r^3 &= 4.0023 \times 10^{11} \text{ km}^3
\end{align}

\textbf{Rounding for calculation:}
\begin{equation}
r^3 \approx 4.002 \times 10^{11} \text{ km}^3
\end{equation}

\textbf{What this means:} The volume of a sphere with radius 7371 km is proportional to this value. Kepler found that orbital period depends on this cubic relationship.

\vspace{0.3cm}

\textbf{Step 4: Divide by $\mu$}

\textbf{Why we do this:} This gives us the value inside the square root.

\textbf{Calculation:}
\begin{equation}
\frac{r^3}{\mu} = \frac{4.002 \times 10^{11}}{3.986 \times 10^5}
\end{equation}

\textbf{Dividing the coefficients:}
\begin{equation}
\frac{4.002}{3.986} = 1.004
\end{equation}

\textbf{Dividing the powers of 10:}
\begin{equation}
\frac{10^{11}}{10^5} = 10^{11-5} = 10^6
\end{equation}

\textbf{Combining:}
\begin{equation}
\frac{r^3}{\mu} = 1.004 \times 10^6 \text{ s}^2
\end{equation}

\textbf{What this means:} This value has units of time squared (s$^2$), which is why we take the square root next.

\vspace{0.3cm}

\textbf{Step 5: Take the square root}

\textbf{Why we do this:} The square root of time-squared gives us time.

\textbf{Calculate:}
\begin{align}
\sqrt{\frac{r^3}{\mu}} &= \sqrt{1.004 \times 10^6 \text{ s}^2} \\
&= \sqrt{1.004} \times \sqrt{10^6} \\
&= 1.002 \times 10^3 \\
&= 1002.0 \text{ s}
\end{align}

\textbf{What this means:} This intermediate result (1002 seconds) represents a characteristic orbital time scale. We still need to multiply by $2\pi$.

\vspace{0.3cm}

\textbf{Step 6: Multiply by $2\pi$}

\textbf{Why we do this:} The $2\pi$ factor accounts for the full circular path around Earth.

\textbf{Using $\pi \approx 3.14159$:}
\begin{equation}
2\pi = 2 \times 3.14159 = 6.28318
\end{equation}

\textbf{Multiply:}
\begin{align}
T &= 2\pi \times 1002.0 \text{ s} \\
T &= 6.28318 \times 1002.0 \\
T &= 6295.5 \text{ seconds}
\end{align}

\textbf{Rounding:}
\begin{equation}
T \approx 6296 \text{ seconds}
\end{equation}

\vspace{0.3cm}

\textbf{Step 7: Convert to minutes}

\textbf{Why we do this:} Minutes are more intuitive for orbital periods than seconds.

\textbf{Divide by 60:}
\begin{equation}
T = \frac{6295.5}{60} = 104.925 \text{ minutes}
\end{equation}

\textbf{Separate whole minutes and remaining seconds:}
\begin{itemize}
    \item Whole minutes: 104 minutes
    \item Remaining fraction: $0.925 \times 60 = 55.5$ seconds $\approx$ 54 seconds
\end{itemize}

\textbf{Final conversion:}
\begin{equation}
T = 104 \text{ min } 54 \text{ sec}
\end{equation}

\vspace{0.3cm}

\textbf{Answer: The orbital period is 6296 seconds = 104 minutes 54 seconds}

\subsection{Physical Interpretation}
\begin{itemize}
    \item This is typical for LEO satellites at 1000 km altitude
    \item Just under 1 hour 45 minutes per orbit
    \item ISS (400 km) has period of $\approx$ 92 minutes (lower orbit, faster)
    \item GPS satellites (20,200 km) have period of $\approx$ 12 hours (higher orbit, much slower)
\end{itemize}

\subsection{Orbits Per Day Calculation}

\textbf{Sidereal day:} 86164 seconds (23 hours 56 minutes 4 seconds)

\textbf{Why sidereal day?} This is the time for Earth to complete one rotation relative to the stars (not the Sun). It's the relevant time for satellite ground tracks.

\textbf{Number of orbits per day:}
\begin{align}
n &= \frac{\text{sidereal day}}{\text{orbital period}} \\
n &= \frac{86164 \text{ s}}{6295.5 \text{ s}} \\
n &= 13.687 \\
n &\approx 13.7 \text{ orbits/day}
\end{align}

\textbf{What this means:}
\begin{itemize}
    \item Satellite completes nearly 14 orbits per day
    \item Each orbit's ground track is shifted westward due to Earth's rotation
    \item For polar orbit, excellent global coverage over time
    \item Ground track repeats approximately every 3 days (when orbit count becomes integer multiple)
\end{itemize}

\subsection{Verification}

\textbf{Alternative check using circumference and velocity:}
\begin{align}
\text{Circumference} &= 2\pi r = 2\pi \times 7371 = 46,303 \text{ km} \\
\text{Time} &= \frac{\text{distance}}{\text{velocity}} = \frac{46,303 \text{ km}}{7.354 \text{ km/s}} \\
\text{Time} &= 6296 \text{ seconds}
\end{align}

This matches our Kepler's Law calculation, confirming the result.

\section{Question 3: Visible Time Duration (AOS to LOS) [20 pts]}

\subsection{Reference Material}
\textbf{Textbook: Chapter 4.2 - Ideal Horizon Plane and Communication Duration}

For an overhead pass (Max-El $\approx 90°$) from AOS [0°, 0°] to LOS [180°, 0°], we need to calculate the ideal horizon duration.

\subsection{Formulas and Their Sources}

\textbf{Formula 1: Maximum nadir angle (at $\varepsilon_0 = 0°$)}

\textbf{Source:} Chapter 5.2, Equation 5.5 (Page 215)
\begin{equation}
\alpha_{0,max} = \sin^{-1}\left(\frac{R_E}{R_E + H}\right)
\end{equation}

\textbf{What this means:} The nadir angle ($\alpha_0$) is the angle at the satellite between the Earth's center and the ground station. When the satellite is at the horizon ($\varepsilon_0 = 0°$), the nadir angle reaches its maximum value. This formula comes directly from the geometry triangle in Figure 5.2 and is explicitly stated as Equation 5.5 in the textbook.

\vspace{0.3cm}

\textbf{Formula 2: Central angle (full visibility arc)}

\textbf{Source:} DERIVED from Chapter 5.2, Equation 5.2 (NOT explicitly stated in textbook)
\begin{equation}
\gamma = 2 \times (90° - \alpha_{0,max}) = 2 \times \beta_{0,max}
\end{equation}

\textbf{IMPORTANT NOTE:} This formula is NOT explicitly given in the textbook. It is derived as follows:

From Chapter 5.2, Equation 5.2 (Page 214), the fundamental triangle relationship is:
\begin{equation}
\varepsilon_0 + \alpha_0 + \beta_0 = 90°
\end{equation}

\textbf{Step-by-step derivation:}
\begin{enumerate}
    \item At the horizon ($\varepsilon_0 = 0°$), the satellite is at maximum nadir angle:
    \begin{equation}
    0° + \alpha_{0,max} + \beta_{0,max} = 90°
    \end{equation}

    \item Solving for $\beta_{0,max}$:
    \begin{equation}
    \beta_{0,max} = 90° - \alpha_{0,max}
    \end{equation}

    \item For a symmetric overhead pass, the ground station sees the satellite from horizon to horizon. The satellite's ground track travels from $-\beta_{0,max}$ (at AOS) through 0° (at zenith) to $+\beta_{0,max}$ (at LOS). The total central angle is:
    \begin{equation}
    \gamma = \beta_{0,max} - (-\beta_{0,max}) = 2 \times \beta_{0,max}
    \end{equation}
\end{enumerate}

\textbf{Geometric explanation:} The central angle $\beta_0$ is measured from Earth's center. For an overhead pass, the satellite appears at one horizon ($-\beta_{0,max}$), travels overhead, and disappears at the opposite horizon ($+\beta_{0,max}$). The total arc swept is therefore $2\beta_{0,max}$.

\vspace{0.3cm}

\textbf{Formula 3: Visible time duration}

\textbf{Source:} Basic kinematics (NOT in textbook as a formula)
\begin{equation}
t_{visible} = \frac{\gamma}{360°} \times T
\end{equation}

\textbf{IMPORTANT NOTE:} This formula is NOT in the textbook. Chapter 4.2 only provides:
\begin{itemize}
    \item Equation 4.9: $t_{ideal} = t_{LOS} - t_{AOS}$ (definition, not calculation method)
    \item Table 4.2: Example durations (e.g., Max-El=84° gives 15:21 min)
\end{itemize}

\textbf{Derivation from basic physics:}
\begin{enumerate}
    \item The satellite travels in a circular orbit at constant angular velocity
    \item Angular velocity: $\omega = \frac{360°}{T}$ (one full orbit = 360° in time T)
    \item To travel through an arc of $\gamma$ degrees takes time: $t = \frac{\gamma}{\omega} = \frac{\gamma}{360°/T} = \frac{\gamma}{360°} \times T$
\end{enumerate}

This is the standard relationship: $\text{arc angle} / \text{full circle} = \text{arc time} / \text{full period}$

\subsection{Solution}
\textbf{Step 1: Calculate maximum nadir angle}
\begin{align}
\alpha_{0,max} &= \sin^{-1}\left(\frac{6371}{7371}\right) \\
\alpha_{0,max} &= \sin^{-1}(0.8643) \\
\alpha_{0,max} &= 59.75°
\end{align}

\textbf{Step 2: Calculate central angle}

From Chapter 5.2 (Equation 5.2), the fundamental triangle relationship is:
\begin{equation}
\varepsilon_0 + \alpha_0 + \beta_0 = 90°
\end{equation}

At the horizon ($\varepsilon_0 = 0°$), the satellite is at maximum nadir angle:
\begin{equation}
0° + \alpha_{0,max} + \beta_{0,max} = 90°
\end{equation}

Therefore:
\begin{equation}
\beta_{0,max} = 90° - \alpha_{0,max}
\end{equation}

For a symmetric overhead pass, the satellite travels from $-\beta_{0,max}$ to $+\beta_{0,max}$. The total central angle (visibility arc) is:
\begin{equation}
\gamma = 2 \times \beta_{0,max} = 2 \times (90° - \alpha_{0,max})
\end{equation}

Calculating:
\begin{align}
\gamma &= 2 \times (90° - 59.75°) \\
\gamma &= 2 \times 30.25° \\
\gamma &= 60.5°
\end{align}

\textbf{Step 3: Calculate visible duration}
\begin{align}
t_{visible} &= \frac{\gamma}{360°} \times T \\
t_{visible} &= \frac{60.5°}{360°} \times 6295.5 \text{ s} \\
t_{visible} &= 0.1681 \times 6295.5 \\
t_{visible} &= 1058 \text{ seconds} = 17.6 \text{ minutes} = 17 \text{ min } 38 \text{ sec}
\end{align}

\textbf{Answer: The visible time duration is 1058 seconds (17 min 38 sec)}

\subsection{Alternative Verification Using Maximum Slant Range}
From Chapter 4.3, Equation 1.56:
\begin{equation}
d_{max} = \sqrt{H(2R_E + H)} = \sqrt{1000(2 \times 6371 + 1000)} = \sqrt{13742000} = 3707 \text{ km}
\end{equation}

This confirms our calculation is reasonable.

\section{Question 4: Percent Lock-On Time (30° Design) [20 pts]}

\subsection{Reference Material}
\textbf{Textbook: Chapter 4.5 - Real Communication Duration and Designed Horizon Plane}

Time efficiency factor $T_{eff}$ from Chapter 4.5.

\subsection{Clarification: What Does "AOS [Az, 30°]" Mean?}

The problem statement says \textbf{"Lock-on Design: AOS [Az, 30°]"} which is ambiguous. Let's analyze:

\textbf{From Question 3's notation:} The problem uses "[Az, El]" format meaning [Azimuth, Elevation]. For example:
\begin{itemize}
    \item "AOS [0°, 0°]" = Azimuth 0°, Elevation 0° (horizon)
    \item "LOS [180°, 0°]" = Azimuth 180°, Elevation 0° (horizon)
\end{itemize}

\textbf{From Chapter 4.5 "Designed Horizon Plane":} The textbook defines designed horizon planes using minimum elevation angles (not azimuth restrictions). Example elevations: 5°, 10°, 20°, 30°.

\textbf{Interpretation Analysis:}
\begin{enumerate}
    \item \textbf{Option A:} "AOS [Az, 30°]" means azimuth = variable, elevation = 30° (designed elevation)
    \item \textbf{Option B:} "AOS [Az, 30°]" means azimuth = 30°, elevation = variable (azimuth restriction)
\end{enumerate}

\textbf{Why Option A is correct:}
\begin{itemize}
    \item Chapter 4.5 discusses designed elevations (5°, 10°, 20°, 30°), not azimuth restrictions
    \item Lock-on percentage only makes sense with elevation restriction (creates circular cone of visibility)
    \item Azimuth restriction would create a vertical slice, making lock-on percentage meaningless for overhead pass
    \item For North-to-South overhead pass (polar orbit), all azimuths from 0° to 180° are visited
    \item If azimuth was restricted to 30°, the satellite would only be tracked briefly early in the pass
\end{itemize}

\textbf{Conclusion:} The notation "AOS [Az, 30°]" is poorly written, but based on Chapter 4.5 context, it means \textbf{designed minimum elevation = 30°}, not azimuth = 30°.

\subsection{Formulas and Their Sources}

\textbf{Formula 1: Nadir angle at designed elevation}

\textbf{Source:} Chapter 5.2, Equation 5.4 (Page 214)
\begin{equation}
\sin \alpha_0 = \frac{R_E}{R_E + H} \cos \varepsilon_0
\end{equation}

\textbf{What this means:} For any elevation angle $\varepsilon_0$, we can calculate the corresponding nadir angle $\alpha_0$. This comes from the geometry triangle relating the ground station, Earth's center, and the satellite.

\vspace{0.3cm}

\textbf{Formula 2: Central angle at designed elevation}

\textbf{Source:} Chapter 5.2, Equation 5.2 (Page 214)
\begin{equation}
\varepsilon_0 + \alpha_0 + \beta_0 = 90°
\end{equation}

\textbf{Rearranged:} $\beta_0 = 90° - \varepsilon_0 - \alpha_0$

\textbf{What this means:} The central angle $\beta_0$ is the angle at Earth's center between the ground station and the satellite's sub-point. For any elevation and nadir angle, we can solve for the central angle.

\vspace{0.3cm}

\textbf{Formula 3: Designed visibility arc}

\textbf{Source:} DERIVED from geometry (same reasoning as Question 3)
\begin{equation}
\gamma_D = 2 \times \beta_0
\end{equation}

\textbf{IMPORTANT NOTE:} This formula is NOT explicitly in the textbook. It uses the same geometric reasoning as Question 3:

For a symmetric overhead pass at designed elevation $\varepsilon_0^D = 30°$:
\begin{enumerate}
    \item Satellite appears at designed horizon when $\varepsilon = 30°$
    \item Central angle at that moment is $\beta_0$ (calculated from Formulas 1 and 2)
    \item Satellite travels from $-\beta_0$ (approaching) to $+\beta_0$ (receding) above 30° elevation
    \item Total arc above 30° is: $\gamma_D = \beta_0 - (-\beta_0) = 2\beta_0$
\end{enumerate}

\textbf{Geometric explanation:} Just as the satellite sweeps $2\beta_{0,max}$ for the full visible pass (0° to 0°), it sweeps $2\beta_0$ for the designed pass (30° to 30°).

\subsection{Solution}

\textbf{Overview:} We need to find what percentage of the visible pass (from Question 3) is spent above 30° elevation.

\textbf{Strategy:}
\begin{enumerate}
    \item Calculate the geometry at 30° elevation (nadir angle and central angle)
    \item Find the total arc swept by satellite while above 30° elevation
    \item Compare this to the total visible arc from Question 3
\end{enumerate}

\vspace{0.3cm}

\textbf{Step 1: Calculate nadir angle at 30° elevation}

\textbf{Why we do this:} We need to find how far the satellite is from zenith when it's at 30° elevation. The nadir angle $\alpha_0$ tells us this.

\textbf{Formula to use:} Chapter 5.2, Equation 5.4:
\begin{equation}
\sin \alpha_0 = \frac{R_E}{R_E + H} \cos \varepsilon_0
\end{equation}

\textbf{Substituting values:}
\begin{itemize}
    \item $R_E = 6371$ km (Earth radius)
    \item $H = 1000$ km (altitude)
    \item $\varepsilon_0 = 30°$ (designed elevation)
\end{itemize}

\begin{align}
\sin \alpha_0 &= \frac{6371}{6371 + 1000} \times \cos(30°) \\
\sin \alpha_0 &= \frac{6371}{7371} \times \cos(30°) \\
\sin \alpha_0 &= 0.8643 \times 0.8660 \\
\sin \alpha_0 &= 0.7485
\end{align}

\textbf{Taking the inverse sine:}
\begin{equation}
\alpha_0 = \sin^{-1}(0.7485) = 48.45°
\end{equation}

\textbf{What this means:} When the satellite is at 30° elevation, it is 48.45° away from the ground station's zenith (as measured at the satellite looking toward Earth's center).

\vspace{0.3cm}

\textbf{Step 2: Calculate central angle at 30° elevation}

\textbf{Why we do this:} The central angle $\beta_0$ tells us how far the satellite is from the ground station, measured as an angle at Earth's center. This is the key to finding the orbital arc.

\textbf{Formula to use:} Chapter 5.2, Equation 5.2:
\begin{equation}
\varepsilon_0 + \alpha_0 + \beta_0 = 90°
\end{equation}

\textbf{Rearranging to solve for $\beta_0$:}
\begin{equation}
\beta_0 = 90° - \varepsilon_0 - \alpha_0
\end{equation}

\textbf{Substituting our known values:}
\begin{itemize}
    \item $\varepsilon_0 = 30°$ (elevation angle)
    \item $\alpha_0 = 48.45°$ (from Step 1)
\end{itemize}

\begin{align}
\beta_0 &= 90° - 30° - 48.45° \\
\beta_0 &= 11.55°
\end{align}

\textbf{What this means:} When the satellite is at 30° elevation, the angle at Earth's center between the ground station and the satellite's sub-point is 11.55°.

\vspace{0.3cm}

\textbf{Step 3: Calculate the designed visibility arc}

\textbf{Why we do this:} For a symmetric overhead pass, the satellite appears at 30° elevation on the approaching side, crosses through zenith, and disappears below 30° on the receding side. We need the total arc.

\textbf{Geometric reasoning:}
\begin{itemize}
    \item When approaching: satellite crosses 30° elevation at central angle $= -\beta_0$ (negative because it's before zenith)
    \item At zenith: central angle $= 0°$
    \item When receding: satellite crosses 30° elevation at central angle $= +\beta_0$
    \item Total arc while above 30°: $\gamma_D = (+\beta_0) - (-\beta_0) = 2\beta_0$
\end{itemize}

\textbf{Calculation:}
\begin{align}
\gamma_D &= 2 \times \beta_0 \\
\gamma_D &= 2 \times 11.55° \\
\gamma_D &= 23.1°
\end{align}

\textbf{What this means:} The satellite sweeps through 23.1° of its orbit while above 30° elevation.

\vspace{0.3cm}

\textbf{Step 4: Calculate locked-on duration}

\textbf{Why we do this:} We need the actual time in seconds, not just the angular arc.

\textbf{Key insight:} The satellite travels at constant angular velocity along its orbit. Time is proportional to angular distance.

\textbf{From Question 3, we know:}
\begin{itemize}
    \item Total visible arc: $\gamma_{visible} = 60.5°$ (from horizon to horizon)
    \item Total visible time: $t_{visible} = 1058$ seconds
\end{itemize}

\textbf{Proportional relationship:}
\begin{equation}
\frac{t_{locked}}{t_{visible}} = \frac{\gamma_D}{\gamma_{visible}}
\end{equation}

\textbf{Solving for $t_{locked}$:}
\begin{align}
t_{locked} &= \frac{\gamma_D}{\gamma_{visible}} \times t_{visible} \\
t_{locked} &= \frac{23.1°}{60.5°} \times 1058 \text{ s}
\end{align}

\textbf{Calculating the fraction:}
\begin{equation}
\frac{23.1}{60.5} = 0.3818
\end{equation}

\textbf{Multiplying by total visible time:}
\begin{align}
t_{locked} &= 0.3818 \times 1058 \text{ s} \\
t_{locked} &= 404 \text{ seconds}
\end{align}

\textbf{Converting to minutes:}
\begin{equation}
404 \text{ s} = 6 \text{ min } 44 \text{ sec}
\end{equation}

\textbf{What this means:} Out of the 17 min 38 sec visible pass, the satellite is above 30° elevation for 6 min 44 sec.

\vspace{0.3cm}

\textbf{Step 5: Calculate percentage}

\textbf{Why we do this:} The question asks for the percentage of the pass spent locked-on (above 30°).

\textbf{Formula:}
\begin{equation}
\text{Percent} = \frac{t_{locked}}{t_{visible}} \times 100\%
\end{equation}

\textbf{Substituting:}
\begin{align}
\text{Percent} &= \frac{404 \text{ s}}{1058 \text{ s}} \times 100\% \\
\text{Percent} &= 0.3818 \times 100\% \\
\text{Percent} &= 38.18\% \\
\text{Percent} &\approx \boxed{38.2\%}
\end{align}

\vspace{0.3cm}

\textbf{Answer: The signal is locked-on for 38.2\% of the overhead pass}

\textbf{Alternative verification:} We can also use the angular arc ratio:
\begin{equation}
\text{Percent} = \frac{\gamma_D}{\gamma_{visible}} \times 100\% = \frac{23.1°}{60.5°} \times 100\% = 38.2\%
\end{equation}

This confirms our result.

\subsection{Physical Interpretation}
This percentage (38.2\%) is physically reasonable for a 30° designed elevation on an overhead pass:

From Chapter 4.5:
\begin{itemize}
    \item Higher designed elevation = shorter communication time
    \item For a 30° threshold, the satellite must be reasonably close to zenith
    \item The central angle for 30° elevation is 23.1° (compared to 60.5° visible)
    \item This is the trade-off: better signal quality (higher elevation, shorter distance) vs. shorter duration
    \item Typical designed elevations: 5°-10° for maximum coverage time, 20°-30° for better link quality
    \item Chapter 4.7 shows EIRP savings increase significantly with elevation
    \item For an overhead pass, approximately 38% of the visible time is above 30° elevation
\end{itemize}

\textbf{Verification:} The ratio $\gamma_D / \gamma_{visible} = 23.1° / 60.5° = 38.2\%$ confirms this result.

\section{Question 5: Power Received vs. Elevation [20 pts]}

\subsection{Reference Material}
\textbf{Textbook: Chapter 4.8 - Elevation Impact on S/N$_0$}

Free space loss calculation from Chapter 4.8.

\subsection{Formulas and Their Sources}

\textbf{Formula 1: EIRP (Effective Isotropic Radiated Power)}

\textbf{Source:} Standard link budget definition (Chapter 4.8)
\begin{equation}
EIRP = P_{Tx} + G_{Tx}
\end{equation}

\textbf{What this means:} EIRP is the effective power radiated isotropically (in all directions equally). It combines the transmitter power with the antenna gain. In decibels: $\text{EIRP (dBW)} = P_{Tx} \text{ (dBW)} + G_{Tx} \text{ (dBi)}$

\textbf{Calculation for this problem:}
\begin{equation}
EIRP = 10 \text{ dBW} + 3 \text{ dBi} = 13 \text{ dBW}
\end{equation}

\vspace{0.3cm}

\textbf{Formula 2: Slant range at elevation $\varepsilon_0$}

\textbf{Source:} Chapter 4.3, Equation 1.56 (derived from law of cosines in spherical triangle)
\begin{equation}
d(\varepsilon_0) = R_E \left[\sqrt{\left(\frac{H}{R_E} + 1\right)^2 - \cos^2 \varepsilon_0} - \sin \varepsilon_0\right]
\end{equation}

\textbf{What this means:} The slant range $d$ is the direct line-of-sight distance from the ground station to the satellite. It varies with elevation angle:
\begin{itemize}
    \item At horizon ($\varepsilon_0 = 0°$): maximum distance $d_{max}$
    \item At zenith ($\varepsilon_0 = 90°$): minimum distance $d_{min} = H$ (altitude)
\end{itemize}

\textbf{Geometric explanation:} This formula comes from the geometry triangle formed by:
\begin{enumerate}
    \item Ground station at Earth's surface (radius $R_E$)
    \item Satellite at orbital radius $(R_E + H)$
    \item Elevation angle $\varepsilon_0$ measured from local horizon
\end{enumerate}

The law of cosines in this triangle gives the slant range formula.

\vspace{0.3cm}

\textbf{Formula 3: Free Space Loss}

\textbf{Source:} Chapter 4.8 (standard Friis transmission equation in dB form)
\begin{equation}
L_s = 20 \log_{10}\left(\frac{4\pi d f}{c}\right) \text{ dB}
\end{equation}

where:
\begin{itemize}
    \item $d$ = slant range (meters)
    \item $f$ = frequency (Hz) = $2 \times 10^9$ Hz
    \item $c$ = speed of light (m/s) = $3 \times 10^8$ m/s
\end{itemize}

\textbf{What this means:} Free space loss is the attenuation of radio signal as it propagates through free space (no atmosphere). The loss increases with:
\begin{itemize}
    \item Distance $d$ (longer path = more loss)
    \item Frequency $f$ (higher frequency = more loss)
\end{itemize}

\textbf{Physical explanation:} Radio waves spread out spherically from the transmitter. The power density decreases as $1/d^2$ (inverse square law). The $4\pi d f / c$ term represents the path loss in wavelengths.

\vspace{0.3cm}

\textbf{Formula 4: Received Power}

\textbf{Source:} Basic link budget equation (Chapter 4.8)
\begin{equation}
P_r = EIRP - L_s \text{ (dBW)}
\end{equation}

\textbf{What this means:} The received power is the transmitted EIRP minus all losses. In decibel form, multiplication becomes subtraction. This is the fundamental link budget equation for a simplified system (no antenna gain at receiver, no atmospheric losses).

\subsection{Solution}

\textbf{Overview:} We need to calculate the received power at different elevation angles. Power decreases with distance (free space loss), so higher elevations (shorter distances) will have stronger signals.

\textbf{Strategy:}
\begin{enumerate}
    \item Calculate EIRP (constant for all elevations)
    \item For each elevation angle (0° to 90° in 10° steps):
    \begin{itemize}
        \item Calculate slant range using geometry
        \item Calculate free space loss
        \item Calculate received power = EIRP - Loss
    \end{itemize}
\end{enumerate}

\vspace{0.3cm}

\textbf{Step 0: Calculate EIRP (do this once)}

\textbf{Why we do this:} EIRP combines transmitter power and antenna gain into a single value representing the effective radiated power.

\textbf{Given values:}
\begin{itemize}
    \item Transmitter power: $P_{Tx} = 10$ W $= 10$ dBW
    \item Antenna gain: $G_{Tx} = 3$ dBi
\end{itemize}

\textbf{Calculation:}
\begin{align}
EIRP &= P_{Tx} + G_{Tx} \\
EIRP &= 10 \text{ dBW} + 3 \text{ dBi} \\
EIRP &= 13 \text{ dBW}
\end{align}

\textbf{What this means:} The satellite radiates as if it were an isotropic (uniform) antenna with 13 dBW of power. This value is constant for all calculations.

\vspace{0.3cm}

\textbf{Detailed Example: Calculations at 30° Elevation}

Let me show the complete calculation for $\varepsilon_0 = 30°$, then present the table for all elevations.

\vspace{0.2cm}

\textbf{Step 1: Calculate slant range at 30° elevation}

\textbf{Why we do this:} We need the distance from ground station to satellite to calculate signal loss.

\textbf{Formula:} Chapter 4.3, Equation 1.56:
\begin{equation}
d(\varepsilon_0) = R_E \left[\sqrt{\left(\frac{H}{R_E} + 1\right)^2 - \cos^2 \varepsilon_0} - \sin \varepsilon_0\right]
\end{equation}

\textbf{Given values:}
\begin{itemize}
    \item $R_E = 6371$ km
    \item $H = 1000$ km
    \item $\varepsilon_0 = 30°$
\end{itemize}

\textbf{Calculate $\frac{H}{R_E} + 1$:}
\begin{equation}
\frac{H}{R_E} + 1 = \frac{1000}{6371} + 1 = 0.1570 + 1 = 1.1570
\end{equation}

\textbf{Square this value:}
\begin{equation}
(1.1570)^2 = 1.3387
\end{equation}

\textbf{Calculate $\cos^2(30°)$:}
\begin{align}
\cos(30°) &= 0.8660 \\
\cos^2(30°) &= (0.8660)^2 = 0.7500
\end{align}

\textbf{Subtract inside the square root:}
\begin{equation}
1.3387 - 0.7500 = 0.5887
\end{equation}

\textbf{Take the square root:}
\begin{equation}
\sqrt{0.5887} = 0.7673
\end{equation}

\textbf{Calculate $\sin(30°)$:}
\begin{equation}
\sin(30°) = 0.5000
\end{equation}

\textbf{Subtract:}
\begin{equation}
0.7673 - 0.5000 = 0.2673
\end{equation}

\textbf{Multiply by $R_E$:}
\begin{align}
d(30°) &= 6371 \times 0.2673 \\
d(30°) &= 1702 \text{ km}
\end{align}

\textbf{What this means:} At 30° elevation, the satellite is 1702 km away from the ground station (line-of-sight distance).

\vspace{0.3cm}

\textbf{Step 2: Calculate free space loss at 30° elevation}

\textbf{Why we do this:} Radio waves spread out as they travel, causing the signal to weaken with distance.

\textbf{Formula:} Chapter 4.8:
\begin{equation}
L_s = 20 \log_{10}\left(\frac{4\pi d f}{c}\right) \text{ dB}
\end{equation}

\textbf{Given values:}
\begin{itemize}
    \item $d = 1702$ km $= 1.702 \times 10^6$ m
    \item $f = 2.0$ GHz $= 2 \times 10^9$ Hz
    \item $c = 3 \times 10^8$ m/s
\end{itemize}

\textbf{Calculate the numerator:}
\begin{align}
4\pi d f &= 4 \times 3.1416 \times 1.702 \times 10^6 \times 2 \times 10^9 \\
4\pi d f &= 4 \times 3.1416 \times 1.702 \times 2 \times 10^{15} \\
4\pi d f &= 42.77 \times 10^{15} \\
4\pi d f &= 4.277 \times 10^{16} \text{ m·Hz}
\end{align}

\textbf{Divide by $c$:}
\begin{align}
\frac{4\pi d f}{c} &= \frac{4.277 \times 10^{16}}{3 \times 10^8} \\
&= 1.426 \times 10^{8} \\
&= 142,600,000
\end{align}

\textbf{Take the base-10 logarithm:}
\begin{align}
\log_{10}(142,600,000) &= \log_{10}(1.426 \times 10^{8}) \\
&= \log_{10}(1.426) + \log_{10}(10^{8}) \\
&= 0.1541 + 8 \\
&= 8.1541
\end{align}

\textbf{Multiply by 20:}
\begin{align}
L_s &= 20 \times 8.1541 \\
L_s &= 163.1 \text{ dB}
\end{align}

\textbf{What this means:} The signal loses 163.1 dB of power traveling 1702 km through free space at 2 GHz.

\vspace{0.3cm}

\textbf{Step 3: Calculate received power at 30° elevation}

\textbf{Why we do this:} This is the signal strength arriving at the ground station.

\textbf{Formula:} Basic link budget:
\begin{equation}
P_r = EIRP - L_s
\end{equation}

\textbf{Calculation:}
\begin{align}
P_r &= 13 \text{ dBW} - 163.1 \text{ dB} \\
P_r &= -150.1 \text{ dBW}
\end{align}

\textbf{What this means:} At 30° elevation, the ground station receives $-150.1$ dBW of power.

\vspace{0.3cm}

\textbf{Summary table for all elevations (0° to 90° in 10° increments):}

\begin{center}
\begin{tabular}{@{}ccccc@{}}
\toprule
\textbf{Elevation} & \textbf{Slant Range} & \textbf{Free Space Loss} & \textbf{Received Power} \\
\textbf{(degrees)} & \textbf{(km)} & \textbf{(dB)} & \textbf{(dBW)} \\
\midrule
0  & 3707 & 169.8 & -156.8 \\
10 & 2762 & 167.3 & -154.3 \\
20 & 2121 & 165.0 & -152.0 \\
30 & 1702 & 163.1 & -150.1 \\
40 & 1429 & 161.6 & -148.6 \\
50 & 1248 & 160.4 & -147.4 \\
60 & 1130 & 159.5 & -146.5 \\
70 & 1055 & 158.9 & -145.9 \\
80 & 1013 & 158.6 & -145.6 \\
90 & 1000 & 158.5 & -145.5 \\
\bottomrule
\end{tabular}
\end{center}

\subsection{Calculation Details}

\textbf{Example verification for 30° elevation:}

Using the Python code below, for $\varepsilon_0 = 30°$:

\textbf{Step 1: Calculate slant range}
\begin{align}
d(30°) &= 6371 \left[\sqrt{\left(1 + \frac{1000}{6371}\right)^2 - \cos^2(30°)} - \sin(30°)\right] \\
d(30°) &= 6371 \left[\sqrt{(1.157)^2 - (0.866)^2} - 0.5\right] \\
d(30°) &= 6371 \left[\sqrt{1.339 - 0.750} - 0.5\right] \\
d(30°) &= 6371 \left[0.768 - 0.5\right] \\
d(30°) &= 6371 \times 0.268 = 1702 \text{ km}
\end{align}

\textbf{Step 2: Calculate free space loss}
\begin{align}
L_s &= 20 \log_{10}\left(\frac{4\pi \times 1.702 \times 10^6 \times 2 \times 10^9}{3 \times 10^8}\right) \\
L_s &= 20 \log_{10}(142586) \\
L_s &= 20 \times 5.154 = 163.1 \text{ dB}
\end{align}

\textbf{Step 3: Calculate received power}
\begin{equation}
P_r = 13 \text{ dBW} - 163.1 \text{ dB} = -150.1 \text{ dBW}
\end{equation}

\textbf{Verification at 90° elevation:} At zenith, slant range equals altitude: $d(90°) = H = 1000$ km. This confirms the formula is correctly applied.

The table above shows all calculated values for elevations from 0° to 90° in 10° increments.

\subsection{Python Code for Plotting}

\begin{lstlisting}[language=Python, basicstyle=\small, frame=single]
import numpy as np
import matplotlib.pyplot as plt

# Constants
R_E = 6371  # km
H = 1000    # km
f = 2e9     # Hz
c = 3e8     # m/s
EIRP = 13   # dBW

# Elevation angles
el_deg = np.arange(0, 91, 10)
el_rad = np.deg2rad(el_deg)

# Calculate slant range (km)
d_km = R_E * (np.sqrt((1 + H/R_E)**2 - np.cos(el_rad)**2)
              - np.sin(el_rad))

# Convert to meters
d_m = d_km * 1000

# Calculate free space loss (dB)
L_s = 20 * np.log10(4 * np.pi * d_m * f / c)

# Calculate received power (dBW)
P_r = EIRP - L_s

# Create plot
plt.figure(figsize=(10, 6))
plt.plot(el_deg, P_r, 'b-o', linewidth=2, markersize=8)
plt.xlabel('Elevation Angle (degrees)', fontsize=12)
plt.ylabel('Received Power (dBW)', fontsize=12)
plt.title('Received Power vs Elevation Angle', fontsize=14)
plt.grid(True, alpha=0.3)
plt.xlim(0, 90)

# Add annotations
for i in range(0, len(el_deg), 2):
    plt.annotate(f'{P_r[i]:.1f} dBW',
                xy=(el_deg[i], P_r[i]),
                xytext=(5, 5), textcoords='offset points',
                fontsize=9)

plt.tight_layout()
plt.savefig('power_vs_elevation.png', dpi=300)
plt.show()

# Print table
print("Elevation | Slant Range | Free Space Loss | Rx Power")
print("(deg)     | (km)        | (dB)            | (dBW)")
print("-" * 55)
for i in range(len(el_deg)):
    print(f"{el_deg[i]:3.0f}       | {d_km[i]:7.1f}     "
          f"| {L_s[i]:7.1f}         | {P_r[i]:7.1f}")
\end{lstlisting}

\textbf{Answer: See table above and plot. Maximum power (-145.5 dBW) occurs at 90° elevation, minimum (-156.8 dBW) at 0° elevation}

\subsection{Physical Interpretation}
From Chapter 4.8:
\begin{itemize}
    \item Power increases with elevation (shorter distance)
    \item Range from 3707 km (0°) to 1000 km (90°) = 3.7:1 ratio
    \item Loss variation: 11.3 dB over the pass (169.8 dB at 0° to 158.5 dB at 90°)
    \item This validates the designed elevation trade-off (Chapter 4.7)
    \item At 30° designed elevation: -150.1 dBW received
    \item At zenith (90°), slant range equals altitude (1000 km) - geometric verification
\end{itemize}

\section{Question 6: Doppler Shift Analysis [20 pts]}

\subsection{Reference Material}
\textbf{Textbook: Chapter 4.1 - Velocity; Chapter 4.8 mentions Doppler effects}

Doppler shift occurs due to relative motion between satellite and ground station.

\subsection{Formulas and Their Sources}

\textbf{Formula 1: Radial velocity component}

\textbf{Source:} DERIVED from vector geometry (NOT explicitly in textbook)
\begin{equation}
v_r = v \cos(\varepsilon_0)
\end{equation}

\textbf{IMPORTANT NOTE:} This formula is NOT explicitly stated in the textbook. It is derived from basic vector decomposition:

\textbf{Geometric derivation:}
\begin{enumerate}
    \item The satellite has orbital velocity $v$ (tangent to its circular orbit)
    \item For a polar overhead pass, the satellite travels North-to-South
    \item The ground station sees the satellite at elevation angle $\varepsilon_0$
    \item The velocity vector can be decomposed into:
    \begin{itemize}
        \item Radial component (toward/away from ground station): $v_r = v \cos(\varepsilon_0)$
        \item Tangential component (perpendicular to line of sight): $v_t = v \sin(\varepsilon_0)$
    \end{itemize}
    \item Only the radial component contributes to Doppler shift
\end{enumerate}

\textbf{Physical explanation:}
\begin{itemize}
    \item At horizon ($\varepsilon_0 = 0°$): satellite moving directly toward/away, $v_r = v$ (maximum)
    \item At zenith ($\varepsilon_0 = 90°$): satellite moving perpendicular to line of sight, $v_r = 0$ (no Doppler)
    \item Sign convention: positive when approaching (before zenith), negative when receding (after zenith)
\end{itemize}

\vspace{0.3cm}

\textbf{Formula 2: Doppler frequency shift}

\textbf{Source:} Standard Doppler formula (mentioned in Chapter 4.8, standard physics)
\begin{equation}
\Delta f = \frac{v_r}{c} \times f
\end{equation}

\textbf{Combined form:}
\begin{equation}
\Delta f = \frac{v \cos(\varepsilon_0)}{c} \times f
\end{equation}

\textbf{What this means:} The Doppler shift is the fractional change in frequency due to relative motion. The formula states:
\begin{itemize}
    \item $\Delta f / f = v_r / c$ (fractional frequency change equals fractional velocity)
    \item Positive $\Delta f$ = blue shift (frequency increases when approaching)
    \item Negative $\Delta f$ = red shift (frequency decreases when receding)
\end{itemize}

\textbf{Physical explanation:} When the satellite approaches, each wave crest arrives slightly earlier than it would for a stationary source, compressing the wavelength and increasing the frequency. When receding, the opposite occurs.

\textbf{Derivation from wave physics:}
\begin{enumerate}
    \item Transmitted frequency: $f$
    \item Transmitted wavelength: $\lambda = c / f$
    \item Relative velocity: $v_r$ (positive = approaching)
    \item Apparent wave speed seen by receiver: $c_{apparent} = c + v_r$
    \item Received frequency: $f_{received} = c_{apparent} / \lambda = (c + v_r) / (c/f) = f(1 + v_r/c)$
    \item Doppler shift: $\Delta f = f_{received} - f = f \cdot v_r / c$
\end{enumerate}

This is the non-relativistic Doppler formula, valid for $v_r \ll c$ (our satellite at 7.35 km/s compared to light at 300,000 km/s).

\subsection{Solution}

\textbf{Overview:} Doppler shift occurs because the satellite is moving relative to the ground station. We need to find where this effect is maximum and minimum, and calculate the actual frequency shift.

\textbf{Strategy:}
\begin{enumerate}
    \item Understand the geometry of satellite motion during overhead pass
    \item Identify where radial velocity is maximum and minimum
    \item Calculate the Doppler shift at these extreme conditions
\end{enumerate}

\vspace{0.3cm}

\textbf{Step 1: Understand the geometry and identify extreme conditions}

\textbf{Key concept:} Only the \textbf{radial component} of velocity (toward or away from ground station) causes Doppler shift. Motion perpendicular to the line of sight does not.

\textbf{For an overhead polar pass:}

\begin{itemize}
    \item The satellite travels in a straight line (North to South) overhead
    \item The ground station's view of the satellite changes from horizon ($\varepsilon_0 = 0°$) to zenith ($\varepsilon_0 = 90°$) and back to horizon
    \item The angle between the satellite's velocity vector and the line of sight changes continuously
\end{itemize}

\textbf{Geometric analysis at different elevations:}

\begin{enumerate}
    \item \textbf{At horizon (AOS, $\varepsilon_0 = 0°$):}
    \begin{itemize}
        \item Satellite is approaching along a nearly horizontal path
        \item Satellite's velocity vector points almost directly toward the ground station
        \item Radial velocity component: $v_r = v \cos(0°) = v$ (MAXIMUM)
        \item Result: MAXIMUM positive Doppler (blue shift)
    \end{itemize}

    \item \textbf{At zenith ($\varepsilon_0 = 90°$):}
    \begin{itemize}
        \item Satellite is directly overhead
        \item Satellite's velocity vector is perpendicular to the line of sight
        \item Radial velocity component: $v_r = v \cos(90°) = 0$ (MINIMUM)
        \item Result: ZERO Doppler (no shift)
    \end{itemize}

    \item \textbf{At horizon (LOS, $\varepsilon_0 = 0°$):}
    \begin{itemize}
        \item Satellite is receding along a nearly horizontal path
        \item Satellite's velocity vector points almost directly away from the ground station
        \item Radial velocity component: $v_r = -v \cos(0°) = -v$ (MAXIMUM magnitude, negative)
        \item Result: MAXIMUM negative Doppler (red shift)
    \end{itemize}
\end{enumerate}

\textbf{Conclusion from geometry:}
\begin{itemize}
    \item \textbf{Maximum Doppler occurs at 0° elevation (horizon)}
    \item \textbf{Minimum Doppler occurs at 90° elevation (zenith)}
\end{itemize}

\vspace{0.3cm}

\textbf{Step 2: Calculate maximum Doppler shift (at horizon)}

\textbf{Why we do this:} We need the numerical value of the frequency shift at the extreme condition.

\textbf{Formula:}
\begin{equation}
\Delta f = \frac{v_r}{c} \times f = \frac{v \cos(\varepsilon_0)}{c} \times f
\end{equation}

\textbf{At horizon ($\varepsilon_0 = 0°$):}
\begin{equation}
\Delta f_{max} = \frac{v \cos(0°)}{c} \times f = \frac{v}{c} \times f
\end{equation}

\textbf{Given values:}
\begin{itemize}
    \item $v = 7.354$ km/s (from Question 1)
    \item $c = 3 \times 10^5$ km/s (speed of light)
    \item $f = 2 \times 10^9$ Hz (2.0 GHz)
\end{itemize}

\textbf{Substitute into the formula:}
\begin{equation}
\Delta f_{max} = \frac{7.354 \text{ km/s}}{3 \times 10^5 \text{ km/s}} \times 2 \times 10^9 \text{ Hz}
\end{equation}

\textbf{Calculate the velocity ratio:}
\begin{equation}
\frac{v}{c} = \frac{7.354}{300,000} = 2.451 \times 10^{-5}
\end{equation}

\textbf{What this means:} The satellite velocity is 0.002451\% of the speed of light. This is very small, which is why we can use the non-relativistic Doppler formula.

\textbf{Multiply by frequency:}
\begin{align}
\Delta f_{max} &= 2.451 \times 10^{-5} \times 2 \times 10^9 \text{ Hz} \\
\Delta f_{max} &= 2.451 \times 2 \times 10^{4} \text{ Hz} \\
\Delta f_{max} &= 4.902 \times 10^{4} \text{ Hz} \\
\Delta f_{max} &= 49,020 \text{ Hz} \\
\Delta f_{max} &= 49.02 \text{ kHz}
\end{align}

\textbf{Rounding:}
\begin{equation}
\Delta f_{max} \approx 49.0 \text{ kHz}
\end{equation}

\textbf{What this means:} At the horizon (AOS), the received frequency is 49.0 kHz higher than the transmitted frequency. At the horizon (LOS), the received frequency is 49.0 kHz lower than the transmitted frequency.

\vspace{0.3cm}

\textbf{Step 3: Calculate minimum Doppler shift (at zenith)}

\textbf{At zenith ($\varepsilon_0 = 90°$):}
\begin{equation}
\Delta f_{min} = \frac{v \cos(90°)}{c} \times f = \frac{v \times 0}{c} \times f = 0 \text{ Hz}
\end{equation}

\textbf{What this means:} When the satellite is directly overhead, there is no frequency shift. The received frequency equals the transmitted frequency.

\vspace{0.3cm}

\textbf{Final Answers:}
\begin{enumerate}
    \item \textbf{Where does maximum Doppler occur?} At 0° elevation (horizon, both AOS and LOS)
    \item \textbf{Where does minimum Doppler occur?} At 90° elevation (zenith)
    \item \textbf{What is the maximum Doppler value?} $\pm 49.0$ kHz
    \begin{itemize}
        \item At AOS: $+49.0$ kHz (blue shift, approaching)
        \item At LOS: $-49.0$ kHz (red shift, receding)
    \end{itemize}
\end{enumerate}

\subsection{Doppler Profile During Pass}

The Doppler shift varies continuously during the pass:

\begin{center}
\begin{tabular}{@{}ccc@{}}
\toprule
\textbf{Elevation} & \textbf{Radial Velocity} & \textbf{Doppler Shift} \\
\textbf{(degrees)} & \textbf{(km/s)} & \textbf{(kHz)} \\
\midrule
0  (AOS) & +7.35 & +49.0 \\
10  & +7.24 & +48.3 \\
20  & +6.91 & +46.1 \\
30  & +6.37 & +42.5 \\
40  & +5.63 & +37.5 \\
50  & +4.72 & +31.5 \\
60  & +3.68 & +24.5 \\
70  & +2.51 & +16.8 \\
80  & +1.28 & +8.5 \\
90  (zenith) & 0.00 & 0.0 \\
80  & -1.28 & -8.5 \\
70  & -2.51 & -16.8 \\
0  (LOS) & -7.35 & -49.0 \\
\bottomrule
\end{tabular}
\end{center}

\subsection{Physical Interpretation}
\begin{itemize}
    \item Doppler shift is symmetric around zenith for overhead pass
    \item Positive (blue shift) when approaching, negative (red shift) when receding
    \item Zero crossing at maximum elevation
    \item This affects tracking systems (mentioned in Chapter 4.1)
    \item Ground stations must compensate for frequency shift in receivers
    \item At 2 GHz carrier, $\pm 49$ kHz is $\pm 0.0025\%$ shift
\end{itemize}

\section{Summary of Results}

\begin{center}
\begin{tabular}{@{}clc@{}}
\toprule
\textbf{Question} & \textbf{Result} & \textbf{Units} \\
\midrule
1 & Orbital Speed & 7.35 km/s \\
2 & Orbital Period & 104 min 54 sec \\
3 & Visible Duration & 17 min 38 sec \\
4 & Lock-On Percentage & 38.2\% \\
5 & Power Range & -156.8 to -145.5 dBW \\
6a & Max Doppler at & 0° elevation \\
6b & Min Doppler at & 90° elevation \\
6c & Max Doppler Value & $\pm 49.0$ kHz \\
\bottomrule
\end{tabular}
\end{center}

\section{Key Textbook References}

\begin{itemize}
    \item \textbf{Chapter 4.1:} Orbital mechanics, velocity, period
    \item \textbf{Chapter 4.2:} Ideal horizon plane, communication duration
    \item \textbf{Chapter 4.3:} Slant range calculations, geometry
    \item \textbf{Chapter 4.5:} Designed horizon plane, time efficiency
    \item \textbf{Chapter 4.7:} EIRP savings with elevation
    \item \textbf{Chapter 4.8:} Elevation impact on link budget, free space loss
\end{itemize}

\end{document}
