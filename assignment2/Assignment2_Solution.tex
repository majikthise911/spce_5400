\documentclass[11pt,letterpaper]{article}
\usepackage[margin=1in]{geometry}
\usepackage{amsmath}
\usepackage{amssymb}
\usepackage{hyperref}
\usepackage{graphicx}
\usepackage{listings}
\usepackage{xcolor}
\usepackage{booktabs}

\title{\textbf{SPCE 5400 Assignment \#2} \\ Ground Station Analysis - Complete Solution}
\author{Topics: Chapters 4 \& 5}
\date{Due: October 15, 2025}

\begin{document}

\maketitle

\section{Problem Statement}

Given a satellite in circular, polar LEO at an altitude of 1000 km, with the following parameters:

\textbf{Satellite Tx Signal:}
\begin{itemize}
    \item Tx power: 10 W (10 dBW)
    \item Frequency: 2.0 GHz
    \item Satellite antenna Gain: 3 dBi
    \item Atmospheric Conditions: No rain, no adverse weather
\end{itemize}

\textbf{Ground Station:}
\begin{itemize}
    \item Location: Pikes Peak, CO [no visible obstructions]
    \item Lock-on Design: AOS [Az, 30°]
\end{itemize}

\section{Constants and Key Equations}

\subsection{Standard Constants}
From Chapter 4.1 and standard references:
\begin{itemize}
    \item Earth radius: $R_E = 6371$ km
    \item Gravitational parameter: $\mu = 3.986 \times 10^5$ km$^3$/s$^2$
    \item Speed of light: $c = 3 \times 10^5$ km/s (or $3 \times 10^8$ m/s)
    \item Frequency: $f = 2.0$ GHz = $2 \times 10^9$ Hz
\end{itemize}

\subsection{Orbital Parameters}
\begin{itemize}
    \item Altitude: $H = 1000$ km
    \item Orbital radius: $r = R_E + H = 6371 + 1000 = 7371$ km
    \item Inclination: $i = 90°$ (polar orbit)
    \item Eccentricity: $e = 0$ (circular orbit)
\end{itemize}

\section{Question 1: Satellite Speed in Orbit [10 pts]}

\subsection{Reference Material}
\textbf{Textbook: Chapter 4.1 - LEO Satellite Tracking Principles}

The orbital velocity for circular orbits is derived from balancing gravitational and centripetal forces.

\subsection{Formula}
From Chapter 4.1:
\begin{equation}
v = \sqrt{\frac{\mu}{r}}
\end{equation}

where:
\begin{itemize}
    \item $v$ = orbital velocity (km/s)
    \item $\mu = 3.986 \times 10^5$ km$^3$/s$^2$ (Earth's gravitational parameter)
    \item $r = R_E + H = 7371$ km (orbital radius)
\end{itemize}

\subsection{Solution}
\begin{align}
v &= \sqrt{\frac{3.986 \times 10^5}{7371}} \\
v &= \sqrt{54.079} \\
v &= 7.354 \text{ km/s}
\end{align}

\textbf{Answer: The satellite speed is 7.35 km/s}

\subsection{Physical Interpretation}
This speed is typical for LEO satellites. For reference:
\begin{itemize}
    \item ISS (400 km): $\approx$ 7.66 km/s
    \item Our satellite (1000 km): 7.35 km/s
    \item Higher altitudes have lower velocities (Kepler's laws)
\end{itemize}

\section{Question 2: Satellite Orbit Period [10 pts]}

\subsection{Reference Material}
\textbf{Textbook: Chapter 4.1 - Orbital Period from Kepler's Third Law}

For circular orbits, the semi-major axis $a = r$.

\subsection{Formula}
From Chapter 4.1:
\begin{equation}
T = 2\pi \sqrt{\frac{a^3}{\mu}} = 2\pi \sqrt{\frac{r^3}{\mu}}
\end{equation}

\subsection{Solution}
\begin{align}
T &= 2\pi \sqrt{\frac{(7371)^3}{3.986 \times 10^5}} \\
T &= 2\pi \sqrt{\frac{4.002 \times 10^{11}}{3.986 \times 10^5}} \\
T &= 2\pi \sqrt{1.004 \times 10^6} \\
T &= 2\pi \times 1002.0 \\
T &= 6295.5 \text{ seconds} \\
T &= 104.9 \text{ minutes} = 104 \text{ min } 54 \text{ sec}
\end{align}

\textbf{Answer: The orbital period is 6296 seconds (104 min 54 sec)}

\subsection{Physical Interpretation}
\begin{itemize}
    \item This is typical for LEO satellites at 1000 km
    \item Sidereal day = 86164 seconds
    \item Number of orbits per day: $n = 86164 / 6295.5 \approx 13.7$ orbits/day
    \item Each orbit covers different ground track due to Earth rotation
\end{itemize}

\section{Question 3: Visible Time Duration (AOS to LOS) [20 pts]}

\subsection{Reference Material}
\textbf{Textbook: Chapter 4.2 - Ideal Horizon Plane and Communication Duration}

For an overhead pass (Max-El $\approx 90°$) from AOS [0°, 0°] to LOS [180°, 0°], we need to calculate the ideal horizon duration.

\subsection{Formulas}
From Chapter 4.2 and 4.3:

\textbf{Maximum nadir angle (at $\varepsilon_0 = 0°$):}
\begin{equation}
\alpha_{0,max} = \sin^{-1}\left(\frac{R_E}{R_E + H}\right)
\end{equation}

\textbf{Central angle (full visibility arc):}
\begin{equation}
\gamma = 2 \times (90° - \alpha_{0,max})
\end{equation}

\textbf{Visible time duration:}
\begin{equation}
t_{visible} = \frac{\gamma}{360°} \times T
\end{equation}

\subsection{Solution}
\textbf{Step 1: Calculate maximum nadir angle}
\begin{align}
\alpha_{0,max} &= \sin^{-1}\left(\frac{6371}{7371}\right) \\
\alpha_{0,max} &= \sin^{-1}(0.8643) \\
\alpha_{0,max} &= 59.75°
\end{align}

\textbf{Step 2: Calculate central angle}

From Chapter 4.2, the visibility arc is defined as the central angle subtended by the satellite's visible path. This is:
\begin{align}
\gamma &= 2 \times (90° - \alpha_{0,max}) \\
\gamma &= 2 \times (90° - 59.75°) \\
\gamma &= 2 \times 30.25° \\
\gamma &= 60.5°
\end{align}

\textbf{Step 3: Calculate visible duration}
\begin{align}
t_{visible} &= \frac{\gamma}{360°} \times T \\
t_{visible} &= \frac{60.5°}{360°} \times 6295.5 \text{ s} \\
t_{visible} &= 0.1681 \times 6295.5 \\
t_{visible} &= 1058 \text{ seconds} = 17.6 \text{ minutes} = 17 \text{ min } 38 \text{ sec}
\end{align}

\textbf{Answer: The visible time duration is 1058 seconds (17 min 38 sec)}

\subsection{Alternative Verification Using Maximum Slant Range}
From Chapter 4.3, Equation 1.56:
\begin{equation}
d_{max} = \sqrt{H(2R_E + H)} = \sqrt{1000(2 \times 6371 + 1000)} = \sqrt{13742000} = 3707 \text{ km}
\end{equation}

This confirms our calculation is reasonable.

\section{Question 4: Percent Lock-On Time (30° Design) [20 pts]}

\subsection{Reference Material}
\textbf{Textbook: Chapter 4.5 - Real Communication Duration and Designed Horizon Plane}

Time efficiency factor $T_{eff}$ from Chapter 4.5.

\subsection{Formulas}
From Chapter 4.5:

For designed elevation $\varepsilon_0^D = 30°$:

\textbf{Nadir angle at designed elevation:}
\begin{equation}
\sin \alpha_0 = \frac{R_E}{R_E + H} \cos \varepsilon_0
\end{equation}

\textbf{Central angle at designed elevation:}
\begin{equation}
\varepsilon_0 + \alpha_0 + \beta_0 = 90°
\end{equation}

Therefore: $\beta_0 = 90° - \varepsilon_0 - \alpha_0$

\textbf{Designed central angle:}
\begin{equation}
\gamma_D = 2 \times \beta_0
\end{equation}

\subsection{Solution}
\textbf{Step 1: Calculate nadir angle at 30° elevation}
\begin{align}
\sin \alpha_0 &= \frac{6371}{7371} \times \cos(30°) \\
\sin \alpha_0 &= 0.8643 \times 0.8660 \\
\sin \alpha_0 &= 0.7485 \\
\alpha_0 &= 48.45°
\end{align}

\textbf{Step 2: Calculate central angle at 30°}
\begin{align}
\beta_0 &= 90° - 30° - 48.45° \\
\beta_0 &= 11.55°
\end{align}

\textbf{Step 3: Calculate designed central angle}
\begin{equation}
\gamma_D = 2 \times 11.55° = 23.1°
\end{equation}

\textbf{Step 4: Calculate locked-on duration}

The satellite moves along its orbital arc at constant angular velocity. The time spent in the lock-on zone is proportional to the angular arc.

For the lock-on zone (above 30° elevation), the satellite travels through a central angle of $\gamma_D = 23.1°$.

But we need to calculate the time based on the \textbf{visible pass duration} (not the full orbital period):

\begin{align}
t_{locked} &= \frac{\gamma_D}{\gamma_{visible}} \times t_{visible} \\
t_{locked} &= \frac{23.1°}{60.5°} \times 1058 \text{ s} \\
t_{locked} &= 0.3818 \times 1058 \\
t_{locked} &= 404 \text{ seconds} = 6 \text{ min } 44 \text{ sec}
\end{align}

\textbf{Step 5: Calculate percentage}
\begin{align}
\text{Percent} &= \frac{t_{locked}}{t_{visible}} \times 100\% \\
\text{Percent} &= \frac{404}{1058} \times 100\% \\
\text{Percent} &= \boxed{38.2\%}
\end{align}

\textbf{Answer: The signal is locked-on for 38.2\% of the overhead pass}

\subsection{Physical Interpretation}
This percentage (38.2\%) is physically reasonable for a 30° designed elevation on an overhead pass:

From Chapter 4.5:
\begin{itemize}
    \item Higher designed elevation = shorter communication time
    \item For a 30° threshold, the satellite must be reasonably close to zenith
    \item The central angle for 30° elevation is 23.1° (compared to 60.5° visible)
    \item This is the trade-off: better signal quality (higher elevation, shorter distance) vs. shorter duration
    \item Typical designed elevations: 5°-10° for maximum coverage time, 20°-30° for better link quality
    \item Chapter 4.7 shows EIRP savings increase significantly with elevation
    \item For an overhead pass, approximately 38% of the visible time is above 30° elevation
\end{itemize}

\textbf{Verification:} The ratio $\gamma_D / \gamma_{visible} = 23.1° / 60.5° = 38.2\%$ confirms this result.

\section{Question 5: Power Received vs. Elevation [20 pts]}

\subsection{Reference Material}
\textbf{Textbook: Chapter 4.8 - Elevation Impact on S/N$_0$}

Free space loss calculation from Chapter 4.8.

\subsection{Formulas}
\textbf{EIRP (Effective Isotropic Radiated Power):}
\begin{equation}
EIRP = P_{Tx} + G_{Tx} = 10 \text{ dBW} + 3 \text{ dBi} = 13 \text{ dBW}
\end{equation}

\textbf{Slant range at elevation $\varepsilon_0$:}
From Chapter 4.3, Equation 1.56:
\begin{equation}
d(\varepsilon_0) = R_E \left[\sqrt{\left(\frac{H}{R_E} + 1\right)^2 - \cos^2 \varepsilon_0} - \sin \varepsilon_0\right]
\end{equation}

\textbf{Free Space Loss:}
From Chapter 4.8:
\begin{equation}
L_s = 20 \log_{10}\left(\frac{4\pi d f}{c}\right) \text{ dB}
\end{equation}

where:
\begin{itemize}
    \item $d$ = slant range (m)
    \item $f$ = frequency (Hz) = $2 \times 10^9$ Hz
    \item $c$ = speed of light (m/s) = $3 \times 10^8$ m/s
\end{itemize}

\textbf{Received Power:}
\begin{equation}
P_r = EIRP - L_s \text{ (dBW)}
\end{equation}

\subsection{Solution}
\textbf{Calculations for each elevation (0° to 90° in 10° increments):}

\begin{center}
\begin{tabular}{@{}ccccc@{}}
\toprule
\textbf{Elevation} & \textbf{Slant Range} & \textbf{Free Space Loss} & \textbf{Received Power} \\
\textbf{(degrees)} & \textbf{(km)} & \textbf{(dB)} & \textbf{(dBW)} \\
\midrule
0  & 3707 & 169.8 & -156.8 \\
10 & 2762 & 167.3 & -154.3 \\
20 & 2121 & 165.0 & -152.0 \\
30 & 1702 & 163.1 & -150.1 \\
40 & 1429 & 161.6 & -148.6 \\
50 & 1248 & 160.4 & -147.4 \\
60 & 1130 & 159.5 & -146.5 \\
70 & 1055 & 158.9 & -145.9 \\
80 & 1013 & 158.6 & -145.6 \\
90 & 1000 & 158.5 & -145.5 \\
\bottomrule
\end{tabular}
\end{center}

\subsection{Calculation Details}

\textbf{Example verification for 30° elevation:}

Using the Python code below, for $\varepsilon_0 = 30°$:

\textbf{Step 1: Calculate slant range}
\begin{align}
d(30°) &= 6371 \left[\sqrt{\left(1 + \frac{1000}{6371}\right)^2 - \cos^2(30°)} - \sin(30°)\right] \\
d(30°) &= 6371 \left[\sqrt{(1.157)^2 - (0.866)^2} - 0.5\right] \\
d(30°) &= 6371 \left[\sqrt{1.339 - 0.750} - 0.5\right] \\
d(30°) &= 6371 \left[0.768 - 0.5\right] \\
d(30°) &= 6371 \times 0.268 = 1702 \text{ km}
\end{align}

\textbf{Step 2: Calculate free space loss}
\begin{align}
L_s &= 20 \log_{10}\left(\frac{4\pi \times 1.702 \times 10^6 \times 2 \times 10^9}{3 \times 10^8}\right) \\
L_s &= 20 \log_{10}(142586) \\
L_s &= 20 \times 5.154 = 163.1 \text{ dB}
\end{align}

\textbf{Step 3: Calculate received power}
\begin{equation}
P_r = 13 \text{ dBW} - 163.1 \text{ dB} = -150.1 \text{ dBW}
\end{equation}

\textbf{Verification at 90° elevation:} At zenith, slant range equals altitude: $d(90°) = H = 1000$ km. This confirms the formula is correctly applied.

The table above shows all calculated values for elevations from 0° to 90° in 10° increments.

\subsection{Python Code for Plotting}

\begin{lstlisting}[language=Python, basicstyle=\small, frame=single]
import numpy as np
import matplotlib.pyplot as plt

# Constants
R_E = 6371  # km
H = 1000    # km
f = 2e9     # Hz
c = 3e8     # m/s
EIRP = 13   # dBW

# Elevation angles
el_deg = np.arange(0, 91, 10)
el_rad = np.deg2rad(el_deg)

# Calculate slant range (km)
d_km = R_E * (np.sqrt((1 + H/R_E)**2 - np.cos(el_rad)**2)
              - np.sin(el_rad))

# Convert to meters
d_m = d_km * 1000

# Calculate free space loss (dB)
L_s = 20 * np.log10(4 * np.pi * d_m * f / c)

# Calculate received power (dBW)
P_r = EIRP - L_s

# Create plot
plt.figure(figsize=(10, 6))
plt.plot(el_deg, P_r, 'b-o', linewidth=2, markersize=8)
plt.xlabel('Elevation Angle (degrees)', fontsize=12)
plt.ylabel('Received Power (dBW)', fontsize=12)
plt.title('Received Power vs Elevation Angle', fontsize=14)
plt.grid(True, alpha=0.3)
plt.xlim(0, 90)

# Add annotations
for i in range(0, len(el_deg), 2):
    plt.annotate(f'{P_r[i]:.1f} dBW',
                xy=(el_deg[i], P_r[i]),
                xytext=(5, 5), textcoords='offset points',
                fontsize=9)

plt.tight_layout()
plt.savefig('power_vs_elevation.png', dpi=300)
plt.show()

# Print table
print("Elevation | Slant Range | Free Space Loss | Rx Power")
print("(deg)     | (km)        | (dB)            | (dBW)")
print("-" * 55)
for i in range(len(el_deg)):
    print(f"{el_deg[i]:3.0f}       | {d_km[i]:7.1f}     "
          f"| {L_s[i]:7.1f}         | {P_r[i]:7.1f}")
\end{lstlisting}

\textbf{Answer: See table above and plot. Maximum power (-145.5 dBW) occurs at 90° elevation, minimum (-156.8 dBW) at 0° elevation}

\subsection{Physical Interpretation}
From Chapter 4.8:
\begin{itemize}
    \item Power increases with elevation (shorter distance)
    \item Range from 3707 km (0°) to 1000 km (90°) = 3.7:1 ratio
    \item Loss variation: 11.3 dB over the pass (169.8 dB at 0° to 158.5 dB at 90°)
    \item This validates the designed elevation trade-off (Chapter 4.7)
    \item At 30° designed elevation: -150.1 dBW received
    \item At zenith (90°), slant range equals altitude (1000 km) - geometric verification
\end{itemize}

\section{Question 6: Doppler Shift Analysis [20 pts]}

\subsection{Reference Material}
\textbf{Textbook: Chapter 4.1 - Velocity; Chapter 4.8 mentions Doppler effects}

Doppler shift occurs due to relative motion between satellite and ground station.

\subsection{Theory and Formulas}
For a satellite passing overhead (polar orbit), the radial velocity component (toward/away from ground station) varies with elevation.

\textbf{Radial velocity:}
For an overhead pass, the geometry shows:
\begin{equation}
v_r = v \cos(\varepsilon_0)
\end{equation}

where $v_r$ is the radial component (positive = approaching, negative = receding).

\textbf{Doppler shift:}
\begin{equation}
\Delta f = \frac{v_r}{c} \times f = \frac{v \cos(\varepsilon_0)}{c} \times f
\end{equation}

\subsection{Solution}
\textbf{Step 1: Identify maximum and minimum Doppler conditions}

From the geometry of an overhead polar pass:
\begin{itemize}
    \item \textbf{Maximum Doppler:} At horizon (El = 0°), when satellite is approaching
    \begin{itemize}
        \item Radial velocity: $v_r = v \cos(0°) = v$ (maximum)
        \item At AOS [0°, 0°]: positive Doppler (blue shift)
        \item At LOS [180°, 0°]: negative Doppler (red shift, same magnitude)
    \end{itemize}
    \item \textbf{Minimum Doppler:} At zenith (El = 90°)
    \begin{itemize}
        \item Radial velocity: $v_r = v \cos(90°) = 0$
        \item Satellite moving perpendicular to line of sight
        \item Doppler = 0 Hz
    \end{itemize}
\end{itemize}

\textbf{Step 2: Calculate maximum Doppler shift}

Using $v = 7.354$ km/s from Question 1:

\begin{align}
\Delta f_{max} &= \frac{v}{c} \times f \\
\Delta f_{max} &= \frac{7.354 \text{ km/s}}{3 \times 10^5 \text{ km/s}} \times 2 \times 10^9 \text{ Hz} \\
\Delta f_{max} &= \frac{7.354}{300000} \times 2 \times 10^9 \\
\Delta f_{max} &= 2.451 \times 10^{-5} \times 2 \times 10^9 \\
\Delta f_{max} &= 49,020 \text{ Hz} = 49.02 \text{ kHz}
\end{align}

\textbf{Answer:}
\begin{itemize}
    \item \textbf{Maximum Doppler} occurs at 0° elevation (horizon, AOS/LOS)
    \item \textbf{Minimum Doppler} occurs at 90° elevation (zenith)
    \item \textbf{Maximum Doppler value:} $\pm 49.0$ kHz
\end{itemize}

\subsection{Doppler Profile During Pass}

The Doppler shift varies continuously during the pass:

\begin{center}
\begin{tabular}{@{}ccc@{}}
\toprule
\textbf{Elevation} & \textbf{Radial Velocity} & \textbf{Doppler Shift} \\
\textbf{(degrees)} & \textbf{(km/s)} & \textbf{(kHz)} \\
\midrule
0  (AOS) & +7.35 & +49.0 \\
10  & +7.24 & +48.3 \\
20  & +6.91 & +46.1 \\
30  & +6.37 & +42.5 \\
40  & +5.63 & +37.5 \\
50  & +4.72 & +31.5 \\
60  & +3.68 & +24.5 \\
70  & +2.51 & +16.8 \\
80  & +1.28 & +8.5 \\
90  (zenith) & 0.00 & 0.0 \\
80  & -1.28 & -8.5 \\
70  & -2.51 & -16.8 \\
0  (LOS) & -7.35 & -49.0 \\
\bottomrule
\end{tabular}
\end{center}

\subsection{Physical Interpretation}
\begin{itemize}
    \item Doppler shift is symmetric around zenith for overhead pass
    \item Positive (blue shift) when approaching, negative (red shift) when receding
    \item Zero crossing at maximum elevation
    \item This affects tracking systems (mentioned in Chapter 4.1)
    \item Ground stations must compensate for frequency shift in receivers
    \item At 2 GHz carrier, $\pm 49$ kHz is $\pm 0.0025\%$ shift
\end{itemize}

\section{Summary of Results}

\begin{center}
\begin{tabular}{@{}clc@{}}
\toprule
\textbf{Question} & \textbf{Result} & \textbf{Units} \\
\midrule
1 & Orbital Speed & 7.35 km/s \\
2 & Orbital Period & 104 min 54 sec \\
3 & Visible Duration & 17 min 38 sec \\
4 & Lock-On Percentage & 38.2\% \\
5 & Power Range & -156.8 to -145.5 dBW \\
6a & Max Doppler at & 0° elevation \\
6b & Min Doppler at & 90° elevation \\
6c & Max Doppler Value & $\pm 49.0$ kHz \\
\bottomrule
\end{tabular}
\end{center}

\section{Key Textbook References}

\begin{itemize}
    \item \textbf{Chapter 4.1:} Orbital mechanics, velocity, period
    \item \textbf{Chapter 4.2:} Ideal horizon plane, communication duration
    \item \textbf{Chapter 4.3:} Slant range calculations, geometry
    \item \textbf{Chapter 4.5:} Designed horizon plane, time efficiency
    \item \textbf{Chapter 4.7:} EIRP savings with elevation
    \item \textbf{Chapter 4.8:} Elevation impact on link budget, free space loss
\end{itemize}

\end{document}
