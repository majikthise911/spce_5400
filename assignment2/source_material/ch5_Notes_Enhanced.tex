\documentclass[11pt,letterpaper]{article}
\usepackage[margin=1in]{geometry}
\usepackage{amsmath}
\usepackage{amssymb}
\usepackage{hyperref}
\usepackage{graphicx}
\usepackage{listings}
\usepackage{xcolor}

\title{\textbf{Chapter 5: LEO Coverage} \\ Enhanced Study Notes}
\author{SPCE 5400 - Ground Station Design}
\date{}

\begin{document}

\maketitle

\tableofcontents
\newpage

\section{LEO Coverage Concept}

\subsection{Overview}
The \textbf{LEO satellite's coverage area} represents the fraction of Earth's surface from where users have visibility and can establish communication with the satellite. Key characteristics:

\begin{itemize}
    \item Coverage area = satellite footprint (circular area on Earth)
    \item Coverage moves as satellite moves
    \item Communication duration varies per pass
    \item Coverage expressed as \% of Earth's surface (typically 1.69\% to 7.95\% for LEO)
\end{itemize}

\subsection{Coverage Area vs Horizon Plane}
\textbf{Important distinction:}

\textbf{Coverage Area:}
\begin{itemize}
    \item Spherical area on ground
    \item Each point has its own horizon plane
    \item Satellite looks DOWN at coverage area
\end{itemize}

\textbf{Ideal/Designed Horizon Plane:}
\begin{itemize}
    \item Virtual flat surface perpendicular to Earth's radius vector
    \item User looks UP at horizon plane
    \item Designed horizon plane parallel to ideal, separated by distance $L_{DHPW}$ (Eq. 4.28)
\end{itemize}

\textit{See textbook Figure 5.1 for 3D visualization of ground station under LEO coverage}

\section{LEO Coverage Geometry}

\subsection{Fundamental Triangle Relationships}

\textit{See textbook Figure 5.2 for complete coverage geometry diagram showing two triangles}

\textbf{Given parameters:}
\begin{itemize}
    \item $H$ = satellite altitude above Earth
    \item $R_E$ = 6371 km (Earth's radius)
    \item $\varepsilon_0$ = elevation angle
    \item $\alpha_0$ = nadir angle
    \item $\beta_0$ = central angle
    \item $d$ = slant range
\end{itemize}

\subsection{Key Equations}

\textbf{Triangle angle relationship:}
\begin{equation}
(\varepsilon_0 + 90) + \alpha_0 + \beta_0 = 180
\end{equation}

\textbf{Simplified (horizon plane perpendicularity):}
\begin{equation}
\varepsilon_0 + \alpha_0 + \beta_0 = 90
\end{equation}

\textbf{Sine theorem application:}
\begin{equation}
\frac{\sin \alpha_0}{R_E} = \frac{\sin(90 + \varepsilon_0)}{R_E + H}
\end{equation}

\textbf{Nadir angle calculation:}
\begin{equation}
\sin \alpha_0 = \frac{R_E}{R_E + H} \cos \varepsilon_0
\end{equation}

\textbf{Maximum nadir angle (full coverage at $\varepsilon_0 = 0$):}
\begin{equation}
\alpha_{0,max} = \sin^{-1}\left(\frac{R_E}{R_E + H}\right)
\end{equation}

\subsection{Coverage Percentage}

\textbf{Definition:}
\begin{equation}
C[\%] = \frac{SAT_{COVERAGE}}{S_{EARTH}}
\end{equation}

Where:
\begin{itemize}
    \item $SAT_{COVERAGE} = 2\pi R_E^2 (1 - \cos\beta_0)$
    \item $S_{EARTH} = 4\pi R_E^2$
\end{itemize}

\textbf{Final coverage formula:}
\begin{equation}
C[\%] = \frac{1}{2}(1 - \cos\beta_0)
\end{equation}

\textbf{Key insight:} Coverage depends on altitude $H$ and elevation angle $\varepsilon_0$, but position on Earth also depends on inclination.

\section{Coverage at Low Elevation}

\subsection{Problem Statement}
\textbf{Idea:} Determine reliable elevation angle that provides safe communication while understanding impact on coverage area width and link budget.

\textbf{Method:} Simulation and mathematical calculations for:
\begin{itemize}
    \item Altitudes: 600 km to 1200 km
    \item Elevation angles: 0° to 10° (steps of 2°)
    \item Calculate $\alpha_0$ from Eq. 5.4, then $\beta_0$ from Eq. 5.2, finally coverage from Eq. 5.7
\end{itemize}

\subsection{Results}

\textbf{Table 5.1: Coverage Areas as Fraction of Earth Area}

\begin{center}
\begin{tabular}{|c|c|c|c|c|}
\hline
\textbf{Elevation ($\varepsilon_0$)} & \textbf{H=600 km} & \textbf{H=800 km} & \textbf{H=1000 km} & \textbf{H=1200 km} \\
\hline
0° & 4.30\% & 5.60\% & 6.80\% & 7.95\% \\
2° & 3.63\% & 4.84\% & 5.95\% & 7.08\% \\
4° & 3.05\% & 4.16\% & 5.21\% & 6.22\% \\
6° & 2.53\% & 3.49\% & 4.54\% & 5.48\% \\
8° & 2.08\% & 3.01\% & 3.91\% & 4.75\% \\
10° & 1.69\% & 2.54\% & 3.38\% & 4.20\% \\
\hline
\end{tabular}
\end{center}

\textit{See textbook Figure 5.3 for 3D graph visualization}

\textit{See textbook Figure 5.4 for orbit simulation at 600km altitude}

\subsection{Conclusions}
\begin{enumerate}
    \item Satellite coverage \textbf{strongly depends} on elevation angle
    \item Largest coverage at $\varepsilon_0 = 0°$, but obstacles require designed elevation (2° to 10°)
    \item Coverage area \textbf{decreases} as elevation increases (for fixed $H$)
    \item Coverage area \textbf{increases} as altitude increases (for fixed $\varepsilon_0$)
    \item LEO satellites at 600-1200 km cover only 1.69\% to 7.95\% of Earth's surface
\end{enumerate}

\section{Coverage Belt}

\subsection{Concept}
As the satellite orbits, its coverage area moves vertically across Earth's surface. The \textbf{coverage belt} is the Earth area swept by LEO satellite's coverage during one complete orbit.

\textit{See textbook Figures 5.5, 5.6, 5.7 for coverage belt visualization and Earth rotation interaction}

\subsection{Coverage Belt Width Calculation}

\textbf{Maximum slant range (at $\varepsilon_0 = 0$):}
\begin{equation}
d_{(\varepsilon_0=0)} = d_{max} = R_E \left[\sqrt{\left(\frac{H + R_E}{R_E}\right)^2 - 1}\right]
\end{equation}

\textbf{Coverage belt width:}
\begin{equation}
D_{BELT} = 2d_{max}
\end{equation}

Coverage belt width depends on:
\begin{itemize}
    \item Altitude $H$ (higher altitude → wider belt)
    \item Elevation angle $\varepsilon_0$ (higher elevation → narrower belt)
\end{itemize}

\subsection{Results}

\textbf{Table 5.2: Coverage Belt Width}

\begin{center}
\begin{tabular}{|c|c|c|c|c|}
\hline
\textbf{Elevation ($\varepsilon_0$)} & \textbf{H=600 km} & \textbf{H=800 km} & \textbf{H=1000 km} & \textbf{H=1200 km} \\
\hline
0° & 5633.0 km & 6579.0 km & 7416.0 km & 8177.8 km \\
2° & 5215.2 km & 6157.2 km & 6991.4 km & 7751.2 km \\
4° & 4824.4 km & 5760.2 km & 6590.6 km & 7347.2 km \\
6° & 4463.0 km & 5386.8 km & 6210.0 km & 6959.8 km \\
8° & 4141.6 km & 5048.8 km & 5859.2 km & 6601.2 km \\
10° & 3857.4 km & 4745.0 km & 5541.8 km & 6273.6 km \\
\hline
\end{tabular}
\end{center}

\textit{See textbook Figures 5.8 and 5.9 for belt width variation visualizations}

\subsection{Conclusions}
\begin{itemize}
    \item Widest belt at $\varepsilon_0 = 0°$ but obstacles require minimum elevation (2° to 10°)
    \item Higher elevation → narrower coverage belt
    \item LEO satellites at 600-1200 km produce coverage belts of 5633 to 8177 km width
\end{itemize}

\section{LEO Global Coverage}

\subsection{Individual vs Global Coverage}

\textbf{Individual Satellite Coverage:}
\begin{itemize}
    \item Single satellite covers only small percentage of Earth (1.69\% to 7.95\%)
    \item Users within footprint can communicate
    \item As satellite moves, coverage moves → users lose communication
    \item Single satellite provides global \textit{access} but not \textit{simultaneous} service
\end{itemize}

\textit{See textbook Figure 5.10 for simulated individual coverage at 800km}

\textbf{Global Coverage via Constellation:}
\begin{itemize}
    \item Multiple satellites organized in constellation
    \item Each contributes individual coverage
    \item \textbf{Interoperability} enables continuous real-time services
    \item Satellites intercommunicate via \textbf{Intersatellite Links (ISL)}
    \item Coverage areas \textbf{overlap} by few degrees for handover
\end{itemize}

\textit{See textbook Figures 5.11 and 5.12 for Iridium constellation and overlapped coverage}

\subsection{Constellation Characteristics}
\textbf{LEO Constellation Definition:}
\begin{itemize}
    \item System of identical LEO satellites
    \item Launched in several orbital planes
    \item Same altitude (single-layer constellation)
    \item Synchronized movement in trajectories relative to Earth
    \item Advanced on-board processing for satellite-to-satellite communication
\end{itemize}

\textbf{Design Considerations:}
\begin{itemize}
    \item Orbit parameters selection
    \item Coverage model
    \item Network connectivity and routing
    \item Handover management policies
    \item Service interruption probability
\end{itemize}

\section{Constellation's Coverage - Starlink Case}

\subsection{Starlink Architecture}
SpaceX's Starlink constellation (as of October 2020):
\begin{itemize}
    \item Nearly 12,000 satellites planned
    \item Organized in 3 orbital shells
    \item Small-dimensioned, lightweight satellites
    \item Goal: ubiquitous broadband internet services
\end{itemize}

\textbf{Three Shells:}
\begin{enumerate}
    \item \textbf{First shell:} 1440 satellites at 550 km altitude (72 planes × 20 satellites)
    \item \textbf{Second shell:} 2825 satellites at 1110 km altitude
    \item \textbf{Third shell:} 7500 satellites at 340 km altitude
\end{enumerate}

\textit{See textbook Figure 5.13 for Starlink satellite train photograph}

\subsection{Coverage Analysis}

\textbf{Method:} Compare three shells using Equations 5.2, 5.4, and 5.7 for:
\begin{itemize}
    \item Full coverage at $\varepsilon_0 = 0°$
    \item Designed elevations: 25°, 30°, 35°, 40°
\end{itemize}

\textbf{Table 5.3: Nadir Angle and Central Angle for Different Elevations}

\begin{center}
\begin{tabular}{|c|cc|cc|cc|}
\hline
\textbf{Elevation} & \multicolumn{2}{c|}{\textbf{Shell 1 (550km)}} & \multicolumn{2}{c|}{\textbf{Shell 2 (1110km)}} & \multicolumn{2}{c|}{\textbf{Shell 3 (340km)}} \\
\textbf{($\varepsilon_0$)} & $\alpha_0$(°) & $\beta_0$(°) & $\alpha_0$(°) & $\beta_0$(°) & $\alpha_0$(°) & $\beta_0$(°) \\
\hline
0° & 66.9 & 23.1 & 58.3 & 31.7 & 71.6 & 18.4 \\
25° & 56.4 & 8.6 & 50.4 & 14.6 & 59.3 & 5.7 \\
30° & 52.8 & 7.2 & 47.5 & 12.5 & 55.2 & 4.8 \\
35° & 48.9 & 6.1 & 44.2 & 10.8 & 51.0 & 4.0 \\
40° & 44.8 & 5.2 & 40.7 & 9.3 & 46.6 & 3.4 \\
\hline
\end{tabular}
\end{center}

\textbf{Table 5.4: Coverage of Starlink Satellites}

\begin{center}
\begin{tabular}{|c|c|c|c|}
\hline
\textbf{Elevation ($\varepsilon_0$)} & \textbf{Shell 1 (550km)} & \textbf{Shell 2 (1110km)} & \textbf{Shell 3 (340km)} \\
\hline
0° & 4.003\% & 7.461\% & 2.55\% \\
25° & 0.560\% & 1.614\% & 0.247\% \\
30° & 0.394\% & 1.185\% & 0.175\% \\
35° & 0.283\% & 0.885\% & 0.121\% \\
40° & 0.206\% & 0.657\% & 0.088\% \\
\hline
\end{tabular}
\end{center}

\subsection{Conclusions}
\begin{itemize}
    \item Very low fraction of Earth covered by single LEO satellite (even at $\varepsilon_0 = 0°$)
    \item Justifies large number of satellites in constellation
    \item Earth's surface = 510 million km²
    \item Example: At 550km altitude and 40° elevation:
    \begin{itemize}
        \item Coverage = 0.00206 × 510M km² = 1.05M km²
        \item Circular area with radius ≈ 580 km
    \end{itemize}
    \item Without overlapping, thousands of satellites needed for continuous global coverage
\end{itemize}

\section{Handover-Takeover Process}

\subsection{Concept}
For continuous real-time services, communication must seamlessly transfer from one satellite to another as satellites move in/out of coverage.

\textbf{Handover-Takeover Process:}
\begin{itemize}
    \item User communicates with Satellite A (within designed horizon plane at 40°)
    \item Satellite A approaches LOS (Loss of Signal) at 40° elevation
    \item Satellite B approaches AOS (Acquisition of Signal) at 40° elevation
    \item Satellites A and B intercommunicate via ISL
    \item User's communication seamlessly transfers from A to B
    \item User experiences no interruption
\end{itemize}

\subsection{Geometrical Interpretation}

\textit{See textbook Figure 5.14 for detailed radar map showing handover geometry}

\textbf{Key Events in Space:}
\begin{itemize}
    \item AOS(0): Acquisition of Signal at ideal horizon (0° elevation)
    \item AOS(40): Acquisition at designed horizon (40° elevation) - communication starts
    \item Max-El: Maximum elevation point (closest approach)
    \item LOS(40): Loss of Signal at designed horizon - communication ends
    \item LOS(0): Loss at ideal horizon
\end{itemize}

\textbf{Example from Figure 5.14 (Table 5.5):}

\textbf{Orbit2:}
\begin{itemize}
    \item AOS2(0) at [155°, 0°] - not locked
    \item AOS2(40) at [220°, 40°] - \textbf{Point A} - locked, range = 809.5 km
    \item Max-El at [310°, 58°] - range = 641.4 km (closest)
    \item LOS2(40) at [345°, 40°] - \textbf{Point B} - unlocked, range = 809.5 km
    \item LOS2(0) at [30°, 0°]
\end{itemize}

\textbf{Orbit3:}
\begin{itemize}
    \item AOS3(0) at [315°, 0°] - not locked
    \item AOS3(40) at [345°, 40°] - \textbf{Point B} - locked, range = 809.5 km
    \item Max-El at [30°, 63°] - range = 611.2 km (closest)
    \item LOS3(40) at [85°, 40°] - \textbf{Point C} - unlocked, range = 809.5 km
    \item LOS3(0) at [125°, 0°]
\end{itemize}

\subsection{Handover Zone Analysis}

\textbf{Point B is the critical handover location:}
\begin{itemize}
    \item Satellite in Orbit2 at 39° elevation (1° before LOS(40))
    \item Distance to user: 827.9 km
    \item Satellite in Orbit3 at 41° elevation (1° after AOS(40))
    \item Distance to user: 800.6 km
    \item Distance between satellites (cosine rule): ≈ 40 km
\end{itemize}

\textbf{This proves:}
\begin{enumerate}
    \item Satellites are close enough to intercommunicate
    \item Coverage overlap of 2° enables smooth handover
    \item User maintains continuous communication from A → B → C
    \item No service interruption detected by user
\end{enumerate}

\subsection{Conclusions}
\begin{itemize}
    \item Handover-takeover is highly coordinated and synchronized
    \item Coverage areas overlap by few degrees for handover zone
    \item Satellites must be adjacent and able to intercommunicate (ISL)
    \item Geometrical confirmation proves continuity of real-time services
    \item Applies to all satellite constellations providing continuous services
    \item Handover policies critical for smooth operation, especially at low elevations
\end{itemize}

\section{Python Implementation Examples}

\subsection{Coverage Percentage Calculator}

\begin{lstlisting}[language=Python, basicstyle=\small, frame=single]
import numpy as np

def calculate_coverage(altitude_km, elevation_deg):
    """
    Calculate LEO satellite coverage percentage.

    Parameters:
    altitude_km: Satellite altitude in km
    elevation_deg: Elevation angle in degrees

    Returns:
    coverage_percent: Coverage as percentage of Earth
    """
    R_E = 6371  # Earth radius in km
    H = altitude_km
    epsilon_0 = np.radians(elevation_deg)

    # Calculate nadir angle (Eq. 5.4)
    sin_alpha_0 = (R_E / (R_E + H)) * np.cos(epsilon_0)
    alpha_0 = np.arcsin(sin_alpha_0)

    # Calculate central angle (Eq. 5.2)
    beta_0 = np.radians(90) - epsilon_0 - alpha_0

    # Calculate coverage (Eq. 5.7)
    coverage_percent = 0.5 * (1 - np.cos(beta_0)) * 100

    return coverage_percent

# Example: Starlink shell 1
altitude = 550  # km
elevation = 40  # degrees
coverage = calculate_coverage(altitude, elevation)
print(f"Coverage: {coverage:.3f}%")
# Output: Coverage: 0.206%
\end{lstlisting}

\subsection{Coverage Belt Width Calculator}

\begin{lstlisting}[language=Python, basicstyle=\small, frame=single]
def calculate_belt_width(altitude_km, elevation_deg):
    """
    Calculate coverage belt width.

    Returns:
    belt_width_km: Width of coverage belt in km
    """
    R_E = 6371
    H = altitude_km
    epsilon_0 = np.radians(elevation_deg)

    # Calculate d_max (Eq. 5.8)
    d_max = R_E * np.sqrt(((H + R_E)/R_E)**2 - 1)

    # For designed elevation, adjust d
    sin_alpha_0 = (R_E / (R_E + H)) * np.cos(epsilon_0)
    alpha_0 = np.arcsin(sin_alpha_0)
    d = R_E * np.sin(alpha_0) / np.cos(epsilon_0)

    # Belt width (Eq. 5.9)
    D_belt = 2 * d

    return D_belt

# Example
belt = calculate_belt_width(800, 0)
print(f"Belt width: {belt:.1f} km")
# Output: Belt width: 6579.0 km
\end{lstlisting}

\subsection{Slant Range Calculator}

\begin{lstlisting}[language=Python, basicstyle=\small, frame=single]
def calculate_slant_range(altitude_km, elevation_deg):
    """
    Calculate slant range between satellite and ground station.
    Uses Eq. 1.56 from Chapter 1.

    Returns:
    slant_range_km: Distance in km
    """
    R_E = 6371
    H = altitude_km
    epsilon = np.radians(elevation_deg)

    # From Eq. 1.56
    d = np.sqrt(R_E**2 + (R_E + H)**2 -
                2*R_E*(R_E + H)*np.cos(np.pi/2 + epsilon))

    return d

# Example: Handover scenario
d_40deg = calculate_slant_range(550, 40)
d_58deg = calculate_slant_range(550, 58)
print(f"Range at 40 deg: {d_40deg:.1f} km")
print(f"Range at 58 deg: {d_58deg:.1f} km")
# Output: Range at 40°: 809.5 km
#         Range at 58°: 641.4 km
\end{lstlisting}

\section{Key Takeaways}

\begin{enumerate}
    \item \textbf{Coverage is Limited:} Single LEO satellites cover only 1.69\% to 7.95\% of Earth
    \item \textbf{Elevation Trade-off:} Lower elevation → larger coverage BUT obstacles require designed elevation (25-40°)
    \item \textbf{Altitude Impact:} Higher altitude → larger coverage area and wider belt
    \item \textbf{Constellations Required:} Thousands of satellites needed for continuous global coverage
    \item \textbf{Handover Critical:} Smooth handover-takeover process ensures service continuity
    \item \textbf{Overlap Necessary:} Coverage areas must overlap by few degrees for handover
    \item \textbf{ISL Essential:} Intersatellite links enable constellation interoperability
    \item \textbf{Starlink Scale:} 12,000 satellites in 3 shells required for global broadband
\end{enumerate}

\section{Connections to Other Chapters}

\begin{itemize}
    \item \textbf{Chapter 1:} Keplerian elements, orbital mechanics fundamentals
    \item \textbf{Chapter 4:} Horizon planes (IHPW, DHPW, LDHPW), tracking principles, slant range equations
    \item \textbf{Chapter 6:} Sun synchronization (affects coverage positioning)
    \item \textbf{Link Budget:} Coverage determines communication windows and ranges for power calculations
\end{itemize}

\end{document}
